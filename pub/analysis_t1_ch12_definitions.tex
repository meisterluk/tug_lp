% definitions.tex
%
% Version: 1.0
%
% Copyright 2010, Lukas Prokop

\documentclass[11pt,a4paper,twocolumn]{article}

% PACKAGES
\usepackage[utf8]{inputenc}
\usepackage{ngerman}
\usepackage{lmodern}

\usepackage{amssymb}
\usepackage{amsmath}
\usepackage{amsthm}
\usepackage{textcomp}
\usepackage{wasysym} % diameter

\usepackage[sf]{titlesec}
\usepackage{fullpage}
\usepackage{booktabs}
\usepackage{fancyhdr}

\usepackage[dvips,pdftex]{geometry}
\usepackage{graphicx}
\usepackage{pstricks}
\usepackage{pst-node}
\usepackage{pst-plot}

\usepackage[unicode,pdfborder={0 0 0}]{hyperref}
\usepackage{boxedminipage}
\usepackage{lastpage}
\usepackage{multicol}
\usepackage{units}
\usepackage{eurosym}
\usepackage{ifthen}
\usepackage{xifthen}
\usepackage{listings}

% Document Configuration
\pagenumbering{arabic}
\pagestyle{plain}
\setcounter{tocdepth}{2}
\parindent0mm
\parskip2mm
\setlength{\unitlength}{1cm}
\renewcommand{\thefootnote}{\arabic{footnote}} % not happy with other ones :-/
\renewcommand{\theequation}{\alph{equation}}

% Aliases
% - general
\newcommand{\mathsymspace}[1]{{\hskip3pt}#1{\hskip3pt}}
\newcommand{\cmd}[1]{{\tt #1}}
\newcommand{\life}[2]{* #1 $\dagger$ #2}
\newcommand{\TODO}{\fbox{\clock \color{blue} TODO}}
\newcommand{\bs}{\textbackslash}

% - math
% --- space
\newcommand{\mathspace}{\hspace{20pt}} % n=1; \mathspace \frac{1}{n}

% --- functions
%\newcommand{\mod}[1]{\bmod{#1}}
\newcommand{\eul}[1]{{\varphi}(#1)}
\newcommand{\vect}[2]{ \left ( \begin{array}{c} #1 \\ #2 \end{array} \right ) }
\newcommand{\ggt}[2]{\textrm{ggT}(#1, #2)}
\newcommand{\divides}{\hskip1pt\mid\hskip5pt}
\newcommand{\ggT}[2]{\text{ggT}(#1, #2)}
%\renewcommand{\binom}[2]{\begin{pmatrix} #1 \\ #2 \end{pmatrix}}
\newcommand{\maximum}[2]{\textrm{max}(#1, #2)}
\newcommand{\minimum}[2]{\textrm{min}(#1, #2)}
\newcommand{\set}[1]{{\Big\{}#1{\Big\}}}

% --- symbols
\newcommand{\ra}{\rightarrow}
\newcommand{\Ra}{\Rightarrow}
\newcommand{\la}{\leftarrow}
\newcommand{\La}{\Leftarrow}
\newcommand{\lra}{\Leftrightarrow}
\newcommand{\average}{\diameter}
\renewcommand{\phi}{\varphi}

% --- complex numbers
\newcommand{\comp}[1]{\overline{#1}}
\newcommand{\conj}[1]{#1 *}

% --- boolean
\newcommand{\andsym}{\mathsymspace{$ \land$}}
\newcommand{\orsym} {\mathsymspace{$ \lor $}}
\newcommand{\notsym}{\mathsymspace{$ \neg $}}
\newcommand{\true}{\hspace{1pt}\text{W}\hspace{1pt}}
\newcommand{\nomathtrue}{\hspace{1pt}W\hspace{1pt}}
\newcommand{\false}{\hspace{3pt}\text{F}\hspace{3pt}}
\newcommand{\nomathfalse}{\hspace{3pt}F\hspace{3pt}}

% - pattern.tex related
% --- general
\newcommand{\linerule}{\hrule{\textwidth}}
\newtheoremstyle{area}
    {6pt}{6pt}{\normalfont}
    {0pt}{\bfseries}{.\hspace{10pt}}{ }{#1}
\theoremstyle{area}
%\theoremstyle{plain}

% --- VO stuff
\newtheorem{mathdef}{Definition}
\newcommand{\question}[1]{\textbf{Q: }#1 \par}
\newcommand{\answer}[1]{\textbf{A: }#1 \par}
\newcommand{\quoded}{
    \parbox{\textwidth}{
        \centering{
            \textsc{
                \small{
                    --- quod erat demonstrandum. ---
                }
            }
        }
    }
}

% --- design, layout
\renewcommand{\epsilon}{\varepsilon}

\title{Definitionen aus der LV \\ Analysis T1}
\author{Lukas Prokop}
\date{\today}

\begin{document}
\maketitle

\section{Logik und Mengenlehre}

\small{Siehe auch ''boolean logic cheatsheet''}

\emph{Ableitungsregel:}
\[
    A \land (A \ra B) \Ra B
\]

\emph{Widerlegungsregel:}
\[
    (\neg B) \land (A \ra B) \Ra \neg A
\]

\emph{Kettenschlussregel:}
\[
    (A \ra B) \land (B \ra C) \Ra (A \ra C)
\]

\emph{Beweis durch Fallunterscheidung}
\[
    (A \lor B) \land (A \ra C) \land (B \ra C) \Ra C
\]

\subsection{Terminologie}

\begin{description}
  \item[$A \Ra B$]
    ''A ist eine hinreichende Bedingung für B''
  \item[$B \Ra A$]
    ''B ist eine notwendige Bedingung für A''
  \item[$\cap$]
    Durchschnitt
  \item[$\cup$]
    Vereinigung
  \item[$\setminus$]
    Differenz
  \item[$A \cap B = \emptyset$]
    A ist elementfremd (disjunkt) zu B
\end{description}

\subsection{Kartesisches Produkt}

Unter dem kartesischen Produkt versteht man alle geordneten Paare mit
a aus A und b aus B.

\[
    A \times B = \set{ (a,b) \mid a \in A, b \in B }
\]

Beispiel:
\[
    \set{a,b} \times \set{x,y,z}
        = \set{(a,x), (a,y), (a,z), (b,x), (b,y), (b,z)}
\]

Es sei \dots
\[
    A^3 = A \times A \times A
\]

$A^3$ bezeichnet man als Tripel. $A^4$ bezeichnet man als Quadrupel.
$A^n$ bezeichnet man als n-Tupel.

\section{Kombinatorik}

Wir definieren \dots

\[
    \# \text{ die Anzahl}
\] \[
    j \text{ als ''Laufindex''}
\]
\begin{equation}
    n! = 1 \cdot 2 \cdot 3 \cdot \ldots \cdot n
\end{equation}
\[
    0! = 1
\]
\begin{equation}
    \sum_{j=0}^n a_j = a_1 + a_2 + \ldots + a_n
\end{equation}
\[
    \sum_{j=n}^n a_j = a_n
\] \[
    m > n: \sum_{j=m}^n a_j = 0
\] \[
    \sum_{j=1}^{n+1} a_j = \sum_{j=1}^{n} a_j + a_{n+1}
\]
\begin{equation}
    \prod_{j=1}^n a_j = a_1 \cdot a_2 \cdot \ldots \cdot a_n
\end{equation}
\[
    m > n: \prod_{j=m}^n a_j = 1
\] \[
    \prod_{j=1}^n (a_j \cdot \lambda)
        = \Big(\prod_{j=1}^n a_j\Big) \cdot \lambda^n
\] \[
    \prod_{j=1}^{n+1} a_j = \prod_{j=1}^{n} a_j \cdot a_{n+1}
\]

Wieviele Möglichkeiten gibt es alle Elemente von M anzuordnen
(Permutation von M)?

\emph{Antwort:} $| M |!$

Wieviele Möglichkeiten gibt es aus n verschiedenen Objekten k
verschiedene Objekte auszuwählen? (Reihenfolge von k berücksichtigt)

\emph{Antwort:} $\prod_{j=0}^{k-1} (n-j)$. Wird auch Variation
$V_k^n$ genannt.

Wieviele Möglichkeiten gibt es aus n verschiedenen Objekten k
verschiedene Objekte auszuwählen? (Reihenfolge von k nicht
berücksichtigt)

\emph{Antwort:} $\frac{n!}{(n-k)! k!} = \binom{n}{k}$. Wird auch
''Binomialkoeffizient'' (gesprochen: n über k) genannt. Oder
''Kombination von n Elementen zur k-ten Klasse'' $C_k^n$.

\begin{equation}
    \frac{n!}{(n-k)! k!} = \binom{n}{k}
\end{equation}
\[
    \binom{n}{0} = \binom{n}{n} = 1
\] \[
    \binom{n}{1} = \binom{n}{n-1} = n
\] \[
    m > n: \binom{n}{m} = 1
\]

\emph{Binomischer Lehrsatz:}

\begin{equation}
    (x + y)^n = \sum_{k=0}^n \binom{n}{k} x^k y^{n-k}
\end{equation}
\[
    \binom{n+1}{k} = \binom{n}{k-1} + \binom{n}{k}
\]

\section{Ringtheorie}

Eine Menge R, formal definiert durch $(R, +, \cdot)$, mit den Operationen
$+$ und $\cdot$ heißt Ring, sofern er die folgenden Kriterien erfüllt:

\begin{itemize}
  \item 0 als neutrales Element bez. Addition
  \item Assoziativgesetz bez. Addition
  \item Kommutativgesetz bez. Addition
  \item Die Zahl selbst negiert als neutrales Element der Addition
  \item Distributivgesetz der Multiplikation bez. Addition
\end{itemize}

Ein \emph{Ring mit Eins} ist ein Ring mit dem zusätzlichen Kritierium
''1 als neutrales Element bez. Multiplikation''. Ein ''kommutativer
Ring'' ist ein Ring mit Eins mit dem zusätzlichen Kriterium
''Kommutativgesetz bez. Multiplikation''.

\section{Zahlenbereiche}

\begin{description}
  % @hfill stupid \LaTeX
  \item[Natürliche Zahlen $\mathbb{N}$] \hfill \\
    $0, 1, 2, 3, 4, \dots$
  \item[Ganze Zahlen $\mathbb{G}$] \hfill \\
    $\dots -3, -2, -1, 0, 1, 2, 3 \dots$
  \item[Rationale Zahlen $\mathbb{R}$] \hfill \\
    $\set{\frac{p}{q} \mid p \in \mathbb{G}, q \in \mathbb{G}^+} $
  \item[Irrationale Zahlen $\mathbb{I}$] \hfill \\
    $\mathbb{R} \setminus \mathbb{Q}$ (Menge der Zahlen, die sich nicht
    als Bruch notieren lassen; zB $\sqrt{2}$, $e$)
  \item[Reele Zahlen $\mathbb{R}$] \hfill \\
    $\mathbb{Q} \cup \mathbb{I}$
  \item[Komplexe Zahlen $\mathbb{C}$] \hfill \\
    $\set{(a + bi) \mid a, b \in \mathbb{R}, i^2 = \sqrt{-1}}$
\end{description}

\begin{equation}
    \mathbb{N} \subset \mathbb{Z} \subset \mathbb{Q}
        \subset \mathbb{R} \subset \mathbb{C}
\end{equation}

\begin{equation}
    (\forall n \in \mathbb{R}: | n | \geq 0)
\end{equation}

Dreiecksungleichung:

\begin{equation}
    | r + s | \leq | r | + | s |
\end{equation}

\section{Folgen}

\begin{quote}
    ''Eine Folge $(a_n)_{n\in\mathbb{N}_0}$ hat höchstens einen Grenzwert''
\end{quote}

Eine Folge $a_n$ ist konvergent zum Grenzwert A, wenn\dots

\begin{equation}
    \forall \epsilon > 0 \exists N \in \mathbb{N}:
        \forall n \geq N : |a_n - A| < \epsilon
\end{equation}

Sei $(a_n)_{n\in\mathbb{N}_0}$ eine Folge, dann heißt $x\in\mathbb{R}$
\emph{Häufungspunkt} der Folge $(a_n)_{n\in\mathbb{N}_0}$, wenn

\begin{equation}
    \forall \epsilon > 0 \forall N \in \mathbb{N} \exists n > N:
        | a_n - x | < \epsilon
\end{equation}

Eine Folge heißt
\begin{itemize}
  \item monoton wachsend: $\forall n \in \mathbb{N}: a_{n+1} \geq a_n$
  \item monoton fallend: $\forall n \in \mathbb{N}: a_{n+1} \leq a_n$
  \item streng monoton wachsend, wenn \\ $\forall n \in \mathbb{N}: a_{n+1} > a_n$
  \item streng monoton fallend, wenn \\ $\forall n \in \mathbb{N}: a_{n+1} < a_n$
\end{itemize}

Eine Nullfolge ist definiert durch: 
\[
    \lim_{n\ra\infty} a_n = 0
\]

\section{Reihen}

\begin{equation}
    (s_n = \sum_{k=0}^n a_k)_{n\in\mathbb{N}_0} = \sum_{n=0}^\infty a_n
\end{equation}

Sei $(a_n)_{n\in\mathbb{N}_0}$ eine Folge. Die \emph{n-te Partialsumme} der
Reihe $\sum_{k=0}^n a_k$ ist definiert durch

\[
    s_n = \sum_{k=0}^n a_k
\]

Eine Menge $A  \subseteq \mathbb{R}$ heißt \emph{beschränkt} (mit $m$
als unterer Schranke und $n$ als oberer Schranke), wenn

\begin{equation}
    \exists m, M \in \mathbb{R} : \forall x \in A : m \leq x \leq M
\end{equation}

Sei $A \subseteq \mathbb{R}$, A beschränkt und $m = \inf{A}$ und
$M = \sup{A}$. Dann gilt:

\begin{equation}
    \forall x \in \mathbb{A} : m \leq x \leq M
\end{equation}
\begin{equation}
    \forall \epsilon > 0 \exists x \in A : x > M - \epsilon
\end{equation}
\begin{equation}
    \forall \epsilon > 0 \exists x \in A : x < m + \epsilon
\end{equation}

Sei $(a_n)_{n\in\mathbb{N}}$ eine beschränkte Folge reeler Zahlen
dann seien:

\begin{equation}
    \limsup{a_n}_{n\ra\infty} = \text{Limes superior}
\end{equation}
\begin{equation}
    \liminf{a_n}_{n\ra\infty} = \text{Limes inferior}
\end{equation}

Eine Reihe $\sum_{n=0}^\infty a_n$ heißt \emph{absolut konvergent},
wenn: $\sum_{n=0}^\infty |a_n|$ konvergiert. Eine konvergente Reihe,
die nicht absolut konvergiert, heißt bedingt konvergent.

\end{document}
