\documentclass{article}

\usepackage[utf8]{inputenc}
\usepackage{amsmath}

\title{Solution to the Monty Hall Problem with the Bayesian Theorem}
\date{\today}
\author{Lukas Prokop}

\newcommand{\Pc}[2]{P(#1\,|\,#2)}

\begin{document}
\maketitle

\section{The problem}
%
In a quizshow, the candidate has to select one of three doors.
Behind two doors there is a goat (an undesirable prize).
Behind the one door there is a car.
At the beginning the candidate tries to guess the door with the car.
Afterwards the moderator Monty Hall opens one of the doors with a goat which is not the candidate's door.
So two doors are left where one of them is the one, the candidate has chosen.
Monty asks the candidate whether he would like to change his mind; he is
free to select the other door.

In general, is it better to change the mind, stick to the first selection
or doesn't it matter?

\section{Bayesian approach}
%
Assumption (one of three cases): The candidate selects door \#1.
\begin{center}
  \begin{tabular}{cc}
    $D_1$ & Car behind door \#1 \\
    $D_2$ & Car behind door \#2 \\
    $D_3$ & Car behind door \#3
  \end{tabular}
\end{center}

\noindent
We define the corresponding probabilities:
\[
    P(D_1) = \frac13  \qquad
    P(D_2) = \frac13  \qquad
    P(D_3) = \frac13
\]

We somehow have to model the behavior of the moderator.
Therefore we introduce a hypothesis $H$ that he opens door \#3.
The probability that he selects door \#3 in favor of door \#2
is $\frac12$, because \#1 is already taken by the candidate:
\[
    P(H) = \frac12
\]

\noindent
Furthermore,
\[
    \Pc{H}{D_1} = \frac12  \qquad
    \Pc{H}{D_2} = 1        \qquad
    \Pc{H}{D_3} = 0
\]

So the probability that he opens door \#3 under the condition that door \#1
contains the car is $\frac12$ (moderator will select door \#2 or door \#1
because \#1 is candidate's door). The probability is $1$ if the car is in
door \#2 (door \#1 is candidate's door and \#2 contains the car).
And if the car is in door \#3, the moderator will never show door \#3.

We can spot some asymmetry here. This point is evidently responsible why
an intuition like ``it does not matter'' might be wrong.

So we will use the Bayes' Theorem:
\[
    \underbrace{\Pc{A}{B}}_{\text{posterior probability}} =
      \frac{\overbrace{\Pc{B}{A}}^{\text{Likelihood}} \cdot
        \overbrace{P(A)}^{\text{priori probability}}
      }{
        \underbrace{P(B)}_{\text{expectedness}}
      }
\]

The important question is: Under the assumption that we have selected door \#1
(our first assumption) and Monty opens door \#3 (our hypothesis or
``expectedness''), what is the probability that the car is behind door \#1
compared to door \#2?
\[
    \Pc{D_1}{H} = \frac{\Pc{H}{D_1} \cdot P(D_1)}{P(H)}
        = \frac{\frac12 \cdot \frac13}{\frac12} = \frac13
\] \[
    \Pc{D_2}{H} = \frac{1 \cdot \frac13}{\frac12} = \frac23
\]

So apparently, the probability that door \#2 contains the car is higher than
for door \#1. Or in other words: in this case the candidate should prefer
to change his mind.

Now we have to evaluate this for other hypotheses (moderator opens door \#2)
and other candidate selections (door \#2 and door \#3). The approach is the
same and not carried out here. But the conclusion keeps the same in all cases:
Being open to change your mind will make you probabilistically win the car.

\end{document}
