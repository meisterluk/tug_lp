% convergence_criteria.tex
%
% Author:  Lukas Prokop <admin@lukas-prokop.at>
% Date:    10.11.04
% Version: alpha
% License: Emailware
%
% Copyright 2010, Lukas Prokop

\documentclass[10pt,twocolumn]{article}

% PACKAGES
\usepackage[utf8]{inputenc}
\usepackage{multicol}
\usepackage[sf]{titlesec}
\usepackage[dvips,pdftex]{geometry}
\usepackage{amssymb}
\usepackage{amsmath}
\usepackage{amsthm}
\usepackage{graphicx}
\usepackage{pstricks}
\usepackage{pst-node}
\usepackage{pst-plot}
\usepackage{boxedminipage}
\usepackage{booktabs}
\usepackage{listings}
\usepackage[unicode]{hyperref}
\usepackage{textcomp}
\usepackage{array}
\usepackage{fullpage}

% Document Configuration
\thispagestyle{empty}
\pagenumbering{arabic} % Alph, alph, arabic, Roman, roman
\parindent0mm
\parskip2mm
\headheight10mm
\headsep10mm
\setlength{\unitlength}{1cm}
\renewcommand{\thefootnote}{\roman{footnote}}
\renewcommand{\theequation}{\alph{equation}}

% Aliases
% - general
\newcommand{\TODO}[1]{}%\fbox{\color{blue} TODO} #1\par}
\newcommand{\note}[1]{\fbox{\textbf{Note:}} #1\par}

% - math
% --- functions
%\newcommand{\mod}[1]{\bmod{#1}}
\newcommand{\eul}[1]{{\varphi}(#1)}
\newcommand{\vect}[2]{ \left ( \begin{array}{c} #1 \\ #2 \end{array} \right ) }
\newcommand{\ggt}[2]{\textrm{ggT}(#1, #2)}
\newcommand{\divides}{\hskip1pt\mid\hskip5pt}
\newcommand{\ggT}[2]{\text{ggT}(#1, #2)}
%\renewcommand{\binom}[2]{\begin{pmatrix} #1 \\ #2 \end{pmatrix}}
\newcommand{\maximum}[2]{\textrm{max}(#1, #2)}
\newcommand{\minimum}[2]{\textrm{min}(#1, #2)}
\newcommand{\set}[1]{\Big\{ #1 \Big\}}

% --- boolean
\renewcommand{\t}{\texttt{\text{T}}\hspace{1pt}}
\newcommand{\nomatht}{\hspace{1pt}T\hspace{1pt}}
\newcommand{\f}{\texttt{\text{F}}\hspace{1pt}}
\newcommand{\nomathf}{\hspace{3pt}F\hspace{1pt}}
\newcommand{\ra}{\rightarrow}
\newcommand{\la}{\leftarrow}
\newcommand{\lra}{\leftrightarrow}
\newcommand{\por}{|}
\newcommand{\pand}{\&}

% - pattern.tex related
% --- general

% --- VO stuff
\newtheorem{mathdef}{Definition}
\newcommand{\quoded}{\textit{--- quod erat demonstrandum. ---}}

% --- design, layout
\renewcommand{\epsilon}{\varepsilon}

\title{Konvergenzkriterien}
\author{Prokop Lukas}

\begin{document}
\maketitle

\section{Konvergenzkriterien für Folgen}

\subsection{Das allgemeine Konvergenzkriterium}

\[
    (\forall \epsilon > 0)(\exists N \in \mathbb{N} :)
        \forall n \geq N : | a_n - A | < \epsilon
\]

\subsection{Cauchy-Kriterium}

Die Folge $(s_n)_{n\in\mathbb{N}}$ konvergiert genau, wenn

\[
    (\forall \epsilon > 0)(\exists N : )(\forall n > N)
        (\forall l \in \mathbb{N}) : | s_{n+l} - s_n | < \epsilon
\]

Dies beschränkt sich jedoch auf $\mathbb{R}$ und $\mathbb{C}$. In
$\mathbb{Q}$ ist nicht jede Cauchy-Folge ebenso konvergent.

\subsection{Hauptsatz über monotone Zahlenfolgen}

Eine nach oben beschränkte monoton wachsende Folge
in $\mathbb{K}$ ist konvergent, ebenso eine nach unten
beschränkte monoton fallende.

\subsection{Beschränktheit, Monotonie und Konvergenz}

Eine monotone und beschränkte Folge ist konvergent.

\subsection{Bolzano-Weierstrauß}

Sei $(a_n)_{n\in\mathbb{N}_0}$ eine beschränkte Folge, dann besitzt
$(a_n)_{n\in\mathbb{N}_0}$ eine konvergente Teilfolge (damit auch eine
Häufungspunkt).

\subsection{Sandwich-Kriterium}

Wenn $\lim_{n\ra\infty} a_n = \lim_{n\ra\infty} c_n = A$ und
$ a_n \leq b_n \leq c_n$ für fast alle $n \in \mathbb{N}$ ist, so ist
auch ($b_n$) konvergent und es ist $\lim_{n\ra\infty} b_n = A$.

\subsection{Nullfolgen}

Wenn $a_n$ eine Nullfolge ist und $b_n$ beschränkt ist, so ist
$(a_n \cdot b_n)$ konvergent und ebenfalls eine Nullfolge:
$\lim_{n\ra\infty} b_n = A$.

\section{Konvergenzkriterien für Reihen}

\subsection{Cauchy-Kriterium}

Die Reihe $\sum_{n=0}^\infty a_n$ konvergiert genau dann, wenn:

\[
    (\forall \epsilon > 0)(\exists N : )(\forall n > N)
        (\forall l \in \mathbb{N}) :
            \Bigg| \sum_{k=n+1}^{n+l} a_k < \epsilon \Bigg|
\]

\subsection{Satz 2.8.3}

Seien $a_n \geq 0$. Die dazugehörige Reihe $s_n = \sum_{n=0}^\infty a_n$
konvergiert genau dann, wenn die Folge der Partialsummen beschränkt ist.

\subsection{Majorantenkriterium}

Sei $\sum_{n=0}^\infty a_n$ eine konvergente Reihe mit positiven Gliedern
und es gelte $\forall n : 0 \leq b_n \leq a_n$, dann konvergiert die
Reihe $\sum_{n=0}^\infty b_n$.

\subsection{Minorantenkriterium}

Sei $a_n \geq 0$ und es divergiere die Reihe $\sum_{n=0}^\infty a_n$.
Sei weiters $b_n \geq a_n$, dann divergiert auch $\sum_{n=0}^\infty b_n$.

\subsection{Verdichtungssatz}

Sei $(a_n)_{n\in\mathbb{N}}$ monoton fallend und positiv. Dann konvergiert
$\sum_{n=0}^\infty a_n$ genau dann, wenn $\sum_{k=0}^\infty 2^k
\cdot a_{2^k}$ konvergiert.

\subsection{Leibniz-Kriterium}

Sei $(a_n)_{n\in\mathbb{N}}$ eine monoton fallende Nullfolge, dann
konvergiert $\sum_{n=0}^\infty (-1)^n \cdot a_n$.

\subsection{Quotientenkriterium}

Sei $(b_n)_{n\in\mathbb{N}}$ eine Folge reeler Zahlen, dann gilt:

\begin{equation}
    \liminf_{n\ra\infty} \frac{| b_{n+1} |}{| b_n |} > 1 \Rightarrow
        \sum_{n=0}^\infty b_n \text{ ist divergent}
\end{equation}
\begin{equation}
    \limsup_{n\ra\infty} \frac{| b_{n+1} |}{| b_n |} < 1 \Rightarrow
        \sum_{n=0}^\infty b_n \text{ ist konvergent}
\end{equation}

Wenn $q = \lim_{n\rightarrow\infty} \frac{| b_{n+1} |}{| b_n |}$ existiert,
dann konvergiert $\sum_{n=0}^\infty b_n$, wenn $q < 1$ und divergiert,
wenn $q > 1$. Bei $q = 1$ ist keine Aussage möglich.

\subsection{Wurzelkriterium}

Sei $(a_n)_{n\in\mathbb{N}}$ eine Folge reeler Zahlen, dann gilt:

\begin{equation}
    \text{Wenn } \limsup_{n\rightarrow\infty} \sqrt[n]{a_n} > 1,
    \text{ dann divergiert die Reihe}
\end{equation}
\begin{equation}
    \text{Wenn } \limsup_{n\rightarrow\infty} \sqrt[n]{a_n} < 1,
    \text{ dann konvergiert die Reihe}
\end{equation}

Wenn $q = \lim_{n\rightarrow\infty} \sqrt[n]{a_n}$ existiert,
so konvergiert die Reihe, wenn $q < 1$, und divergiert, wenn
$q > 1$. Bei $q = 1$ ist keine Aussage möglich.

\section{Referenzfolgen}

\[
    \lim_{n\ra\infty} \frac{1}{n} = 0
\] \[
    \lim_{n\ra\infty} \sqrt[n]{a} = 1
\] \[
    \lim_{n\ra\infty} \sqrt[n]{n} = 1
\] \[
    \lim_{n\ra\infty} \sqrt[n]{n!} = \infty
\] \[
    \lim_{n\ra\infty} \frac{a^n}{n!} = 0
\] \[
    \lim_{n\ra\infty} \Big(1 + \frac{1}{n}\Big)^n = e
\]

\section{Referenzreihen}

Mit $k \in \mathbb{N}^+$

\[
    \sum_{n=1}^\infty \frac{1}{k \cdot n} \text{ ist divergent}
\] \[
    \sum_{n=1}^\infty \frac{1}{n^\alpha} \text{ ist}
    \begin{cases}
        \text{konvergent} & \text{ für } \alpha > 1 \\
        \text{divergent} & \text{ für } \alpha \leq 1 \\
    \end{cases}
\]

\end{document}
