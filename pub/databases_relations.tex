\documentclass[a4paper,twocolumn]{article}

\usepackage[utf8x]{inputenc}
\usepackage{graphicx}
\usepackage{amssymb}

% TODO: complete relational algebra
% TODO: SQL syntax tutorial and differences between postgresql, mysql,
%       sqlite, ...

\author{Lukas Prokop}
\title{Databases and relations}
\date{26th of January 2012}

\begin{document}
\maketitle
\tableofcontents

\section{Databases}

\textbf{Note.} Based on lectures ''Databases 1'', ''Databases 2'', ''New
Information systems'' and ''Object-oriented analysis and design'' I visited.

\subsection{Information Systems}

An \emph{information model} describes the problem (ie. application) domain,
which gets reduced in the software development process to the domain of
interest. This domain captures only relevant parts of the application and
provides a focus.

A \textbf{data object} is a collection of attribute-value pairs that model
an individual entity. A database is an information model of the real world.

\subsection{Database design}

Database design consists of three phases:
%
\begin{itemize}
  \item Conceptual design (DBMS independent)
  \item Logical design (transactions independent)
  \item Physical design (schema for a particular DBMS and transactions)
\end{itemize}

Conceptual design relates in some kind to the requirement analysis phase in
a software development process. The goals of a conceptual design are:
%
\begin{itemize}
  \item to identify data objects
  \item to identify all attributes of data objects
  \item to identify relationships between data objects
  \item to identify integrity constraints
\end{itemize}

Quality attributes for the conceptual design are:
%
\begin{itemize}
  \item integrability
  \item learnability and memorability
  \item completness
  \item visualability
\end{itemize}

\subsection{DataBase Management System (DBMS)}

A DBMS' goal is to keep its data integrity while applying operations to create,
modify and access data in the database. A DBMS represents a data model and
separates a database structure (\emph{database schema}) from its contents
(\emph{database}).

\begin{itemize}
  \item A \emph{data model} consists of a collection of data structure types,
        a collection of operators and rules and a collection of general
        integrity rules (implicitly and explicitly).
  \item \emph{Database transactions} have to be
        ACID (atomic, consistent, isolated and durable).
        Transactions are a logical unit of work and are an application-specified
        sequence of data manipulation operations to avoid inconsistent state
        after an error half-way through the sequence. Therefore a transaction
        is either executed in its entirety or totally aborted.
\end{itemize}

\subsection{Database Languages}

\begin{description}
  \item[Data Description Language (DDL)]
    A DDL is used to define database schemas. Its syntax is a collection of
    statements for the description of data types.
  \item[Data Manipulation Language (DML)]
    A DML allows the user to manipulate data objects and relationships
    between them. It's a collection of operators that can be applied to
    data objects. Manipulation includes insertion, deletion and modification
    of tuples.
  \item[Query Language (QL)]
    A QL uses queries as parameters to select and project data set from
    one or several relations.
\end{description}

There are different approaches to use DDL/DML/QLs:
%
\begin{description}
  \item[Self-Contained Language]
    For example SQL. It's command-oriented, english-like
    and can become complex.
  \item[Embedded Host Language]
    For example C++. The program uses a DML (eg. SQL) from an application
    backend perspective to manipulate the data. A possible approach is to
    make simple queries (QL) to the database and resolve its relationships
    within the application programming language.
\end{description}

\subsection{Roles in a DBMS}

\begin{description}
  \item[Database Administrator (DBA)]
    A DBA is capable of using a DDL to create and maintain a
    consistent set of database schemas to satisfy the needs of the
    problem domain.
  \item[Application Developers (AD)]
    An AD uses an interface from the application to communicate to the
    database using a DML. He develops specific functions around the database
    and manipulates the data.
  \item[End User] The end user does not get in contact with the structure of
    a database and only gets an user interface to enter data. The
    application resolves that data to DML queries.
\end{description}

\subsection{Protections}

Data protection is crucial to an application.
%
\begin{description}
  \item[Physical Protection]
    It addresses protection against physical loss of data.
    This loss can be caused by natural disasters, theft or accidental damage
    to equipment.

  \item[Operational Protection]
    Operational Protection prevents human errors on the databases'
    integrity. Integrity constraints specified by a DBA in the database
    schema can help to protect from these errors.

  \item[Authorisational Protection]
    Read and write access to the databases should only be given to
    authorised users to ensure confidentiality and correctness.
\end{description}

\subsection{Data Manipulation Facilities}

Can be done using a DML.
%
\begin{itemize}
  \item insert or add new tuples into the relation
  \item delete or erase existing tuples from the relation
  \item modify or change data in an existing relation.
        assignment or computation (readed, computed, stored)
\end{itemize}

\section{Relational Database Model}

\subsection{Terminology}

\begin{description}
  \item[attributes]
    distinct columns in a relation.
    Each attribute has a associated domain.
  \item[cardinality]
    number of rows in a table
  \item[degree]
    number of columns in a table
  \item[domain]
    a set of different values with the same property types (''data type'').
    The NULL value is possible if not explicitly stated otherwise.
  \item[relation]
    a table (visual view) with a unique name.
    a two dimensional, inhomogeneous matrix (mathematical view).
    The ordering of tuples in a relation is important.
  \item[schema]
    schema of a relation refers to the permanent characteristics
    of a relation
  \item[tuple]
    the set of values in a row.
    Those values are instances of the attribute domain.
\end{description}

\subsection{Keys}

The definition of a relation includes the definition of attributes,
domains, a primary key and additional constraints. The definition
of primary keys per relation allows us to simply refer entries in
other relations.

\begin{description}
  \item \emph{Keys} are unique.
  \item Keys shall be kept small to reduce the foreign key table
        in terms of storage size. Furthermore a small key allows
        fast and efficient index lookups.
  \item \emph{Superkeys} are attributes identifying a tuple uniquely.
  \item A \emph{candidate key} is a superkey with a minimal set of
        attributes and candidate for a primary key.
  \item One of the candidate keys is declared as \emph{primary key}
        and satisfies the relation constraint and does not contain a
        NULL value.
  \item A \emph{foreign key} is an attribute which holds the primary key of
        another relation.
\end{description}

\paragraph{Possible notation \\}

\textless relation name\textgreater:
    (\textless attribute name 1\textgreater,
     \textless attribute2\textgreater,
     \textless attribute3\textgreater
    ) \\
relation: (\textbf{primary key}, attribute,
    \textit{foreign key}) \\
Customer: (\textbf{cid}, name, age, \textit{userRightsId})

\subsection{Constraints}

\subsubsection{Integrity constraints}

\emph{Integrity} is the reliability through imposition of constraints
during data manipulation (data integrity).

\begin{itemize}
  \item implicit
    \begin{description}
      \item[entity integrity] A primary key cannot be set to NULL.
      \item[referential integrity] A foreign key can have only two
        possible values either the relevant primary key or NULL value.
    \end{description}
  \item explicit (examples are given)
    \begin{description}
      \item[domain] upper/lower limits for integer values
      \item[tuple] if attr1 $==$ x then y has to be $< 50$
      \item[relation] no duplicates as primary keys
    \end{description}
\end{itemize}

\section{Normalization}

\subsection{Anomalies}

\begin{description}
  \item[Update anomaly] Data inconsistency or loss of data integrity
    can arise from data redundancy or repetition and partial update
  \item[Insertion anomaly]
    Data cannot be added because some other data is absent.
  \item[Deletion anomaly]
    Data maybe unintentionally lost through the deletion of other data.
\end{description}

\subsection{Terminology}

\begin{itemize}
  \item A \emph{determinant} is an attribute or a set of non-redunant
    attributes which can act as a unique identifier of (a set of)
    another attribute of a given value. So if two tuples have the same
    value for $A$, it's required to have the same value for $B$ in both
    tuples.
    $A \rightarrow B$ (''A determines B'' or ''A is a determinant of B'').
    The opposite is $A \nrightarrow B$ (''A does not determine B'').
  \item If $A$ determines $B$ then $B$ is \emph{functionally dependent}
    on $A$. \emph{Full functional dependency} is given if $A$ determines
    $B$ and no subset of $A$ determines $B$.
  \item The relation $(A \rightarrow B \land B \rightarrow C) \Rightarrow
    A \rightarrow C$ describes transitivity. Transitivity is called
    \emph{indirect dependency}.
\end{itemize}

\subsection{1. Normal Form (1NF)}

A relation in 1NF has to be free of repeating groups. Even though this
definition is considered differently by various researchers, it basically
means that no value contains more than one value and there are no duplicate
tuples. Codd also makes references to the concept of atomicity:
%
\begin{quote}
    values in the domains on which each relation is defined are required to
    be atomic with respect to the DBMS.
\end{quote}

\subsection{2. Normal Form (2NF)}

A table is in 2NF if it is in 1NF and no attribute---that is not a part
of any candidate key of the table---is dependent on any proper subset of
any candidate key of the table.

So 2NF can be achieved by taking a relation in 1NF and eliminating all
partial key dependencies.

\subsection{3. Normal Form (3NF)}

A table is in 3NF if it is in 2NF and every non-key attribute is fully and
directly dependent on the primary key. So 3NF can be achieved by taking a
relation in 2NF and eliminating all indirect dependencies on the primary
key.

\section{Relational Algebra}
%
\begin{table}[h]
  \begin{center}
    \begin{tabular}{rl}
      $\sigma$ & selection \\
      $\pi$    & projection \\
      $\cup$   & union \\
      $-$      & difference \\
      $\times$ & cartesian product \\
      $\rho$   & rename \\
      $\bowtie$& join \\
      $\ltimes$& left semi join \\
      $\rtimes$& right semi join \\
      $\cap$   & intersection \\
      $\div$   & division \\
    \end{tabular}
  \end{center}
\end{table}
%
Relation Algebra is a procedural relationally complete language.
A language that can define any relation definable in relational calculus
is relationally complete. Using this algebra we perform set operations
on relations.

\subsection{Selection $\sigma$}

A selection is a unary operation to select tuples from relations.

\begin{verbatim}
select <source relation name>
where <predicate>
giving <result relation name>
\end{verbatim}

\begin{itemize}
  \item \texttt{predicate} might be any kind of propositional formula
        using the logical operators AND, OR or NOT. NOT has to be prefixed;
        other operators get infixed.
        The precedence is defined as followed: NOT, AND, OR.
  \item intension (schema) stays the same.
  \item the result is a horizontal subset of source.
  \item \texttt{giving} clause can be ommitted to use the result inline.
\end{itemize}

\subsection{Projection $\pi$}

The projection is an operation to reduce a relation containing containing
only specified columns. All duplicate result tuples will be removed.

\begin{verbatim}
project <source relation name>
over <list of attribute names>
giving <result relation name>
\end{verbatim}

The list of attribute names shall be given as comma separated list.

\subsection{Rename $\rho$}

The rename operation renames an attribute of a relation to a new name.

\begin{verbatim}
Rename <attribute name> <target name>
\end{verbatim}

\subsection{Natural join $\bowtie$}

A natural join of two tables combines those tables i.e. extends the tuples of
one relation with their counterparts. Two tuples (of each table) will be
contactenated if their shared (specified) attributes match. It ommits tuples
which don't have an associated tuple in the other relation.

\begin{verbatim}
join <first relation> AND <second relation>
over <list of attribute names>
giving <result relation name>
\end{verbatim}

The list of attribute names shall be given as comma separated list.

\subsection{Division $\div$}

A division builds sets of tuples that are identical in their non-shared
attributes and filters sets, which don't contain all corresponding shared
values of the divisor. All shared values are dropped and resulting
duplicates get removed.

\begin{verbatim}
divide <source relation name>
by <divisor relation name>
giving <result relation name>
\end{verbatim}

The divisor relation has to share attributes with the source relation.

\subsection{Union $\cup$}

The union operation combines two tuple-sets and ommits duplicates.
Two tuples are union-compatible, if they have identical
intensions.

\begin{verbatim}
<first source relation>
union <second source relation>
giving <result relation>
\end{verbatim}

\subsection{Intersection $\cap$}

The intersection outputs all tuples which are in both source relations.

\begin{verbatim}
<first source relation>
intersect <second source relation>
giving <result relation>
\end{verbatim}

\subsection{Difference $\setminus$}

This operation outputs all tuples from the first relation,
which are not also in the second relation.

\begin{verbatim}
<first source relation>
minus <second source relation>
giving <result relation>
\end{verbatim}

\subsection{NULL values}

Any comparison involving NULL returns FALSE.

\begin{itemize}
  \item So in a selection operation tuples with NULL values
        in condition attributes are omitted in the resulting
        relation.
  \item In a projection, the resulting tuples with NULL values
        are not considered to be duplicates and therefore do
        not get reduced.
  \item A join operation with tuples containing NULL in a
        shared attribute excludes those tuple from the list
        of tuples to concatenate.
  \item A division operation will fail if there is a NULL value
        in the divisor (because not tuple set will match). The
        result is an empty relation.
  \item For other operations (union, intersection, difference)
        tuples containing NULL are considered to be different.
\end{itemize}

\subsection{Relational algebra example}

For example, the SQL SELECT statement combines the operators Selection,
Projection, Join and the Cartesian Product of relational algebra.

\begin{verbatim}
select Movie
where m_title = 'x' or m_title = 'y'
giving XYMovie;

project XYMovie
over m_id
giving XYMovieId;

divide Role
by XYMovieId
giving XYRole;

join XYRole and Human
over h_id
giving XYHuman;

project XYHuman
over h_first_name, h_last_name
giving Result;
\end{verbatim}

\section{Glossary}

\begin{description}
  \item[RDBMS] Relational Database Management System
  \item[DBMS] Database Management System
  \item[DDL] Data Description Language
  \item[DML] Data Manipulation Language
  \item[DBA] Database Administrator
\end{description}

\section{Entity Relationships}

\subsection{Entities}

An entity in ER is an instance of a more abstract idea. It's a
''thing'' or an ''object'' in the real world that is distinguishable
from other objects.

\begin{itemize}
  \item Each entity occurrence is presented as a number of attributes
  \item Each attribute has a name and domain.
  \item Domains specify the set of permitted values for attributes.
\end{itemize}

An \emph{entity type} is a set of similar entity occurrences.
The term ''similar'' can be perceived as ''presenting with different
values of the same domains''. The model always operates on ''entities''
and ''entity types''; so those terms are used synonymously.

\subsection{Relationships}

Relationships describe associations between an arbitrary number of
entity occurences. Each relationship has a name and may have zero
or more attributes. A relationship type is a set of similar
relationship instances. The term ''similar'' can be simply seen
as ''relationshipy between occurrences of the same entity types,
having one and the same name and set of attributes''.
Accordingly the model operates on ''relationship types'' so the terms
are used synonymously. Pragmatically, we understand the term
''relationship'' as a description of common properties of a set
of relationship instances (name of the relationship, members, attributes).

\subsection{Constraints}

Cardinality constraints define the number of entity occurrences to which
an other entity occurrence can be associated via a relationship instance.

\subsubsection{Cardinality constraints}

A one-to-one relationship can be visualized by a simple line between
the first entity instance and the relationship and another line between
the relationship and the second instance.

A one-to-many or many-to-many relationship can be visualized by using a
split symbol at the end of the line before it ends into the corresponding
instance. The line has to be split into three straight lines.

\subsection{Terminology}

\begin{description}
  \item[relationships with total participation]
        occurrences of a certain entity must obligatory participate
        in such relationships (for example, an employee has to
        participate in a relationship ''works\_at'').
  \item[relationships with partial participation]
        occurrences of a certain entity may optionally participate
        in such relationships (for example, an employee may
        participate in a relationship ''is\_manager'').
  \item[weak entity]
        The primary key of an entity is defined by an attribute
        and an attribute of a foreign entity.
\end{description}

\subsection{Entity Relationship Diagrams}

Entity Relationship Diagrams (ER) are a graphical notation for the
Entity Relationship Model introduced before.

\begin{itemize}
  \item Entities are represented by rectangulars.
  \item Attributes are represented by ellipses with a line to its
        associated entity
  \item Relationships are represented by diamonds.
  \item Attributes of a relationship are represented by ellipses with
        a line to its associated entity
  \item Primary key attributes are underlined in the diagram.
  \item Total participation of an entity in a relationship is
        represented by double (thick) lines connection
        the relationship and the entity.
\end{itemize}


\end{document}
