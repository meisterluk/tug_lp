\input{../../pattern.tex}
\meta{Fragenausarbeitung \\ Einführung in das Studium der Informatik}{10.11.21}{Prof. Bischof Horst}
\maketitle

\section{Die Universität}

\subsection{Mit welchen Personen an der TU Graz haben Sie zu tun?}

\begin{itemize}
  \item Tutoren
  \item Professoren
  \item Studienkommission
  \item Studiendekan
  \item Basisgruppe BIS
  \item ÖH
  \item Studienservice
  \item Vizerektor
  \item Rektor
\end{itemize}

\subsection{Welche Studien können Sie an der Fakultät für Informatik absolvieren?}

\begin{itemize}
  \item Informatik
  \item Softwareentwicklung-Wirtschaft
  \item Lehramt Informatik und Informatikmanagement
  \item Telematik
\end{itemize}

\section{Das Studium}

\subsection{Welche LVs aus dem Bereich der Programmierung sind im Studium
    Informatik enthalten?}

\begin{itemize}
  \item Einführung in die strukturierte Programmierung
  \item Softwareentwicklung Praktikum
  \item Softwareparadigmen
  \item Entwurf und Entwicklung großer Systeme
  \item \dots
\end{itemize}

\subsection{Rechnen Sie die ECTS in den Arbeitsaufwand eines durchschnittlichen
    Arbeitnehmers um\dots}

1 ECTS = 25 Arbeitsstunden. 30 ECTS pro Semester. 6 Semester. Entsprechen
also 180 ECTS. 180 ECTS mal 25 Arbeitsstunden = 1500 Arbeitsstunden.

Die entspricht etwa einem durchschnittlichen Arbeitnehmer.

\subsection{Worin unterscheidet sich das Lehramts-Studium Informatik
    vom Studium Informatik?}

\begin{itemize}
  \item 10-semestrig statt 6+4-semestrig
  \item Integrationsaufgaben in Schulen in punkto Informatik
  \item Kenntnisse der Psychologie (Entwicklungspsychologie)
  \item Kenntnisse aus Allgemeiner Didaktik und Fachdidaktik
  \item Fähigkeit zur Kommunikation und zum Dialog mit Schülern,
        Lehrern, Eltern, Behörden
  \item Kenntnisse aus Schulrecht
\end{itemize}

\section{Die Informatik}

\subsection{Definieren Sie den Begriff ''Informatik''}

\begin{quote}
    Informatik ist die Wissenschaft von der systematischen und
    automatisierten Verarbeitung von Information. Sie erforscht
    grundlegende Verfahrensweisen der
    Informationsverarbeitung und allgemeine Methoden ihrer
    Anwendung in den verschiedenen Bereichen. Für diese
    Aufgaben wendet die Informatik vorwiegend formale und
    ingenieurmäßig orientierte Techniken an. \\
    -- Gesellschaft für Informatik
\end{quote}

\begin{quote}
    Die Informatik ist eine Wissenschaft, die sich mit dem
    Interpretieren, Strukturieren und der Verarbeitung von
    Information als Daten befasst und Werkzeuge zur automatisierten
    Bearbeitung bereitstellt.
\end{quote}

\begin{quote}
    Informatique ist die Behandlung von Information mit rationalen Mitteln \\
    -- Akademie Francaise

    Rationale Mittel nach René Déscartes:
    \begin{itemize}
      \item Dasjenige gilt als wahr, was so klar ist, dass kein Zweifel
        bleibt.
      \item Größere Probleme sind in kleinere aufzuspalten.
      \item Es ist immer vom Einfachen zum Zusammengesetzten hin zu
        argumentieren.
      \item Das Werk muss einer abschließenden Prüfung unterworfen
        werden.
    \end{itemize}
\end{quote}

\begin{quote}
    Informatik -- das ist die Faszination, sich die Welt der
    Informationen und des symbolisierten Wissens zu
    erschließen und dienstbar zu machen \\
    Aus dem ''Positionspapier'' der Gesellschaft für Informatik
\end{quote}

\begin{quote}
    Informatik befasst sich mit der Transformation von Information in
    Systemen.
\end{quote}

\subsection{Definieren Sie den Begriff Information}

\begin{quote}
    Information nennen wir den abstrakten Gehalt (''Bedeutungsinhalt'',
    ''Semantik'') einer Nachricht. Die äußere Form der Darstellung
    nennen wir die Repräsentation (konkrete Form der Nachricht). \\

    Claude Shannon
\end{quote}

\subsection{Definieren Sie den Begriff Entropie}

Unter Entropie versteht man ein ''Maß für die Unordnung in einem System''.
Höhere Entropie bedeutet eine höhere Wahrscheinlichkeit, dass ein
gewisses Ereignis $p_i$ eintritt. Diese Definition ist frei nach
Claude Shannon, der mit seinem Werk ''A Mathematical Theory of Communication''
diesen Begriff geprägt hat.

\subsection{Was versteht man unter der ''Augmented Reality''?}

Unter ''Augmented Reality'' versteht man die Erweiterung der Realität
mittels technischer Geräte. Man entwickelt Schnittstellen zum realen
Leben, um die Wahrnehmung zu unterstützen und insbesonders zusätzliche
Informationen zu liefern. Als Beispiel seien Autoerkennung von bekannten
Bauwerken (und automatisches Finden eines passenden Wikipedia-Artikels)
genannt oder das automatische Erkennen von Entfernung auf Basis reiner
Bildinformation.

\subsection{Es wurde der Begriff ''ubiquitäres Rechnen'' verwendet. Geben
    Sie eine Definition}

Unter Ubiquitous Computing bezeichnet die Allgegenwärtigkeit der
rechnergestützten Informationsverarbeitung.

In der Vorlesung wurde
die technologische Entwicklung der Geräte seit den 60ern betrachtet.
Während früher große Mainframes von zahlreichen technisch versierten
Personen gemeinsam verwendet wurden, führte der Trend bis (etwa) zur
Jahrtausendwende hin zum Personal Computer; einem Gerät, welches nur
von einer Person verwendet wird. Mit dem Beginn des ubiquitären Rechnens
besitzt eine einzelne Person mehrere Geräte, die für verschiedene Aufgaben
zur Verfügung stehen (als größtes Beispiel sei hier das Handy/Smartphone
genannt, weitere sind Handhelds, Netbooks, Chips). In der Zukunft
erwartet man einen verstärkten Einsatz des Cloud Computing (Verwendung
von Servern zur Datenspeicherung und Datenverarbeitung).

\subsection{Aus welchen Perspektiven lässt sich die Informatik betrachten?}

\begin{itemize}
  \item Informatik als Grundlagenwissenschaft
  \item Informatik als Ingenieursdisziplin
  \item Informatik als Experimentalwissenschaft
\end{itemize}

\subsection{Grenzen Sie die Informatik gegenüber anderen Wissenschaften ab}

\begin{itemize}
  \item Elektrotechnik -- Hardware und Systeme
  \item Technische Mathematik -- Algorithmen und Theorie
  \item Telematik -- Generalisten zwischen Informatik und Elektrotechnik
  \item Informatik -- Software und Theorie zwischen Mathematik und
    Elektrotechnik
\end{itemize}

Informatik verwendet Techniken und Prinzipien der Mathematik zur
Entwicklung (der Software) informationsverarbeitender Systeme. Informatik
verwendet Techniken und Prinzipien der Elektrotechnik zur Entwicklung
(der Hardware) informationsverarbeitender Systeme. Wie die Mathematik
dient die Informatik den Geistes- und Naturwissenschaften als
Hilfswissenschaft, die Methoden und Werkzeuge (meist in Form von
Computerprogrammen) bereitstellt. Zwischen anderen Wissenschaften und der
Informatik bestehen Wechselwirkungen in dem Sinn, dass Methoden der einen
auf die andere angewandt werden können und umgekehrt. Eine Folge dieses
Sachverhalts ist die Einrichtung gemeinsamer Studiengänge
(''Bindestrich-Informatiken'').

\subsection{Nennen Sie 5 Fragen, mit denen sich die Informatik grundlegend
    beschäftigt}

\begin{itemize}
  \item Was ist berechenbar?
  \item Was ist Information?
  \item Was ist Denken?
  \item Wie können wir komplexe Systeme verstehen?
  \item Wie können wir komplexe Systeme bauen?
\end{itemize}

\subsection{Nennen Sie Bahnbrechende Ergebnisse der Informatik}

\begin{itemize}
  \item Rechnerarchitektur (von Mainframes bis RFID-Chips)
  \item Internet
  \item Neudefinition des Begriffs Suche
  \item Mensch oder Maschine (''Deep Blue vs Kasparov'')
  \item Das Internet der Dinge (ambient intelligence)
  \item Mixed Reality
  \item PRIMES is in P
  \item Asymmetrische Kryptographie
  \item Polynomial-time factoring on quantum computer
\end{itemize}

\subsection{Teilen Sie die Informatik in Teildisziplinen auf}

\begin{itemize}
  \item Technische Informatik: Rechnerarchitektur, Hardware,
    Mikroprozessoren, Netzwerke, \dots
  \item Praktische Informatik: Programmierung, Compilerbau, Algorithmen,
    \dots
  \item Theoretische Informatik: Codierungstheorie, Automatentheorie,
    Berechenbarkeitstheorie, Komplexitätstheorie, \dots
  \item Angewandte Informatik: Anwendungssoftware, Buchhaltung,
    Verfahrenstechnik, Multimedia, Simulation, \dots
\end{itemize}

\subsection{Zählen Sie 3 bekannte Algorithmen auf}

\begin{enumerate}
  \item Euklidischer Algorithmus (größter gemeinsamer Teiler)
  \item Sieb des Eratosthenes (Primzahlberechnung)
  \item Mergesort (Liste von Zahlen sortieren)
  \item Quicksort (Liste von Zahlen sortieren)
  \item Bubblesort (Liste von Zahlen sortieren)
  \item Dijkstra-Algorithmus (Graphenwegsuche)
  \item A*-Algorithmus (Graphenwegsuche)
  \item Vigenère-Algorithmus (Verschlüsselung)
  \item ENIGMA (Verschlüsselung)
  \item Binäre Exponentation (Vereinfachung von Quadrierung
    und Multiplikation)
\end{enumerate}

\subsection{Beschreiben Sie die Grundzüge einer Von Neumann Architektur}

\begin{quote}
    Programme sind Daten \\
    -- John von Neumann
\end{quote}

Der \emph{Rechenkern} (CPU) führt einzelne Instruktionen durch und
arbeitet damit Aufgaben schrittweise ab. Er hat die Möglichkeit
Daten in den \emph{Arbeitsspeicher} abzulegen und auf diese mittels
Speicheradressen wieder zuzugreifen. Im Arbeitsspeicher befindet sich sowohl
das laufende Betriebsystem, die Applikationen wie auch die Daten derer.
Über \emph{Peripheriegeräte} werden Daten in den Rechner eingeschleust, die
Benutzerinteraktion ermöglichen. Das \emph{Bussystem} verbindet die
genannten Komponenten und bietet Kommunikationswege.

\subsection{Zählen Sie 5 bekannte Informatiker auf}

\begin{itemize}
  \item Charles Babbage (Verschlüsselung, Analytical Engine)
  \item Ada Lovelace (erste Programmiererin)
  \item Konrad Zuse (baut den ersten funktionsfähigen programmegesteuerten
    Rechenautomat)
  \item 
\end{itemize}

\subsection{Zählen Sie 10 Programmiersprachen auf}

\begin{itemize}
  \item Ada
  \item Fortran
  \item LISP
  \item Algol
  \item Simula
  \item C
  \item Smalltalk
  \item Haskell
  \item ML
  \item C++
  \item Java
  \item C\#
  \item python
  \item perl
  \item ruby
\end{itemize}

\subsection{Welche Informatiker kennst du und in welchen Bereichen
    haben sie gearbeitet?}

\emph{Erste Generation}

\begin{itemize}
  \item John von Neumann (1903--1957) (Rechnerarchitektur)
  \item Kurt Gödel (1906--1978) (Logik)
  \item Alan Turing (1912--1945) (Automatentheorie, Berechenbarkeitstheorie,
    Komplexitätstheorie, Brechen von Kryptosystemen)
  \item George Boole (1815--1864) (Logik)
  \item Claude Shannon (1916--2001) (Begründer der Informationstheorie)
  \item Joseph Weizenbaum (1923--2008) (Linguistik, Künstliche Intelligenz)
  \item John Backus (1924--2007) (Formalisierung, Programmiersprachen)
  \item Edsger Dijkstra (1930--2002) (Algorithmen, Berechenbarkeitstheorie)
\end{itemize}

\emph{Zweite Generation}

\begin{itemize}
  \item Marvin Minsky (* 1927) (Künstliche Intelligenz)
  \item Noam Chomsky (* 1928) (Sprachen, Formalisierung)
  \item Niklaus Wirth (* 1934) (Programmiersprachen, Pascal)
  \item Donald E. Knuth (* 1938) (Algorithmen, Datenstrukturen,
    Berechenbarkeitstheorie)
  \item Dennis Ritchie (* 1941) (UNIX, C)
  \item Brian Kernighan (* 1942) (UNIX, C)
  \item Ken Thompson (* 1943) (UNIX, C)
  \item Andrew S. Tanenbaum (* 1944) (Betriebsysteme, UNIX)
  \item Whitfield Diffie (* 1944) (Kryptologie)
  \item Ron Rivest (* 1947) (Kryptologie)
  \item Bjarne Stroustrup (* 1950) (C++)
  \item Adi Shamir (* 1952) (Kryptologie)
  \item Richard Stallman (* 1953) (GNU)
  \item Tim Berners-Lee (* 1955) (Begründer des WWW)
  \item Bruce Schneier (* 1963) (Kryptologie, IT Security)
  \item Paul Graham (* 1964) (Programmiersprachen)
  \item Linus Torvalds (* 1969) (Linux)
\end{itemize}

\subsection{Nennen Sie 3 Punkte, die sie (nach Vorstellung des Studiums
    im Zuge der LV) am Studium überraschen\dots}

\end{document}
