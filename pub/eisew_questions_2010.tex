\input{../../pattern.tex}
\newcommand{\uncertain}{\fbox{Vorsicht!}
    Lösung nicht in Materialien vorhanden. Kein Gewähr auf Korrektheit}
\meta{VO Einführung in die Softwareentwicklung-Wirtschaft Prüfungsfragenausarbeitung}{10.11.02}{Prof. Leberl Franz}
\maketitle

\emph{Materialien:} Mitschriften, Test von 2008, Folien, Test vom 3. Nov 2010

\section{Prof. Leberl Franz}

\subsection{Definiere den Begriff ''Softwareentwicklung''}

Unter Softwareentwicklung verstehen wir das saubere, zielgerichtete, wohl
geplante und qualitätsgesicherte Schreiben von großen Computerprogrammen in
Teams. Verschiedene Teilbereiche (Modellbildung und Design von Systemen,
Implementation und Testen von Software) befassen sich damit diese Ziele zu
erreichen, um den Anforderungen (Robustheit, Wiederverwendbarkeit,
Erweiterbarkeit, Testbarkeit, Zukunftssicherheit) gerecht zu werden.

\subsection{Wofür stehen die Akronyme IEEE und ACM?
    Was können Sie dazu sagen?}

\begin{description}
  \item[IEEE] Institute of Electrical and Electronics Engineers.
  \item[ACM] Association for Computing Machinery.
\end{description}

Die beiden Gesellschaften verwalten eine Liste für Teilgebiete der
elektronischen Datenverarbeitung. Universitäten weltweit versuchen
mit ihren Lehrplänen diese Teilgebiete mittels Lehrveranstaltungen
für Informatikstudien abzudecken.

\subsection{Welches sind die 5 Studien, welche von IEEE und ACM als Zugänge
    zum Computing definiert wurden?}

\begin{itemize}
  \item Computer Science
  \item Computer Engineering
  \item Software Engineering
  \item Information Systems
  \item Information Technology
\end{itemize}

\subsection{Welche LVs aus dem Bereich Wirtschaft sind im Studium
    Softwareentwicklung-Wirtschaft vertreten?}

\begin{itemize}
  \item Buchhaltung und Bilanzierung
  \item Steuerrecht
  \item Kosten- und Erfolgsrechnung
  \item Betriebswirtschaftslehre
  \item Betriebssoziologie
  \item (Bürgerliches Recht und Unternehmensrecht)
\end{itemize}

\subsection{Bill Gates sprach vom ''First Digital Decade''. Was bezeichnet
    er damit?}

Computer wurden für den Standardanwender zugänglicher. Die
Breitbandanschlüsse nehmen zu. Handys und digitale Photographie
sind Dinge des Standardnutzers. Die Wege Musik zu hören haben
sich ebenso wesentlich geändert. Bill Gates geht weiters davon
aus, dass sämtlicher Konsum von Unterhaltung bald nur mehr ein
Softwarethema sein wird. Diese erste Phase bezeichnet er als
''First Digital Decade''.

Die zweite Digitale Decade wird sich (nach Bill Gates) eher auf
die Interaktion von Menschen in Netzen konzentrieren. Dies soll
benutzerzentriert erfolgen. Die Applikationen werden dabei im
Sinne des Cloud Computing verwirklicht und laufen nicht mehr lokal.

\url{http://www.microsoft.com/presspass/ofnote/10-29digitaldecade.mspx}

\section{Dr. Havemann Sven}

\subsection{Zählen Sie bitte zumindest 2 verschiedene Betriebssysteme auf,
    von denen Sie gehört haben.}

\begin{itemize}
  \item Die Windows-Serie
  \item GNU/Linux
  \item MacOS / Darwin
  \item Android
  \item iOS
  \item Symbian OS
  \item MS DOS
  \item ReactOS
  \item SunOS / Solaris
  \item OpenBSD
  \item FreeBSD
  \item NetBSD
  \item GNU Hurd
  \item Minix
  \item BeOS
\end{itemize}

\subsection{Was macht ein ''Betriebsystem?''}

Das Betriebsystem nimmt die Position einer Schnittstelle zwischen Hardware
und Applikationen ein. Es handelt sich um eine erweiterte Maschine, die
''unangenehme Details versteckt, die ausgeführt werden müssen''.
Es ''präsentiert seinem Benutzer eine virtuelle Maschine, die einfacher
verwendbar ist''. Es ist in einem Schichtenmodell aufgebaut, welches die
gewünschte Funktionalität auf Basis einer wesentlich niedrigeren Ebene
aufbaut. Es ist ein Programm (wie jedes andere -- wird kompiliert,
gelinkt und exekutiert) mit dem Sonderstatus einer einzelnen Kopie und wird
über Service-Aufrufe aktiviert. Es verwaltet Ressourcen, abstrahiert die
Hardware, stellt höhere Funktionalität den Applikationen zur Verfügung
und kontrolliert den Ablauf der einzelnen Applikationen.

\subsection{Welche fundamentalen Aufgaben erfüllt ein Prozessor?}

Er hat den sogenannten Fetch-Execute-Algorithmus implementiert.
Er holt den Befehl aus dem Speicher (''fetch'') und exekutiert ihn
entsprechend dem Befehlscode. Eventuell müssen entsprechende Operanden
ebenso geladen werden. Das Ergebnis der Operation wird in den Hauptspeicher
geschrieben und der Befehlszähler wird erhöht.

\begin{quote}
  Programme sind auch nur Daten! \\
  -- Designphilosophie der von Neumann Architektur
\end{quote}

\subsection{Welche Lehrveranstaltungen im Zusammenhang mit Datenbanken
    werden Sie im Zuge des Studiums SEW besuchen? Welche Inhalte werden
    Ihnen dabei vermittelt?}

\begin{itemize}
  \item \emph{Datenbanken 1:} Umgang mit Datenbanksysteme, Datenmodelle,
    Relationale Datenbanken, Relationale Algebra, SQL, QBE, Recovery
  \item \emph{Datenbanken 2:} Data Manipulation Language, Host Programming
    Language, Java Servlets
\end{itemize}

\section{Prof. Elsholtz Christian}

\subsection{Welche LVs aus dem Bereich der Mathematik sind Teil des
    Studiums Softwareentwicklung-Wirtschaft?}

\begin{itemize}
  \item Analysis T1 (1. Semester)
  \item Diskrete Mathematik (2. Semester)
  \item Numerisches Rechnen und lineare Algebra (3. Semester)
  \item Wahrscheinlichkeitstheorie für Informatikstudien (3. Semester)
  \item Statistik für Informatikstudien (3. Semester)
\end{itemize}

\subsection{Was versteht man unter dem Begriff eines Graphen?
    Welche sind die Kernelemente der Graphentheorie?}

Ein Graph ist eine Menge von Knotenpunkten, die durch sogenannte Kanten
miteinander verbunden sind und Beziehungen zwischen den Punkten darstellen.
Als Beispiel sei das Haus des Nikolaus genannt, welches aus einem Quadrat
mit 2 Diagonalen und 1 Dach besteht. In einem Graph lässt jetzt beschreiben
welche Punkte mit einem anderen Punkt direkt verbunden sind.

Die Graphentheorie wird verwendet, um größere Probleme auf einfache
Graphenprobleme herunterzubrechen und über diese eine Aussage treffen
zu können.

In der Vorlesung wurde die Adjazenzmatrix (Methode zur Speicherung
von Graphen in Computern) vorgestellt. Die vorgestellten
Probleme TSP (Problem des Handlungsreisenden) und Spannbaumproblem
(Problem des minimalen Spannbaums) lassen sich auch als Graphen anschaulich
darstellen.

\subsection{Erläutern Sie Optimierungstechniken zur Ermittlung von $3^{1000}$}

\fbox{Notiz:} In den Vorlesungsunterlagen ist das Verfahren sehr
    minimalistisch beschrieben. Es wird hier die Variante von
    Wikipedia\footnote{
    \url{http://en.wikipedia.org/wiki/Exponentiation_by_squaring}}
    vorgestellt.

Die naheliegendste Variante besteht daraus 1000mal eine Multiplikation
mit 2 durchzuführen. Dies ist jedoch absolut ineffizient in Differenz
zu diesem Algorithmus:

Notiere die Zahl 1000 binär ($1111101000_2$). Eine Eins an der Position
$x$ steht für $3^{2^x}$. Ersetze 0 durch S und 1 durch SM. Entferne
ein führendes SM-Paar, da $1000 > 0$. Man nehme jetzt die Basis her und
fasse S als entsprechendes Square (Quadrieren) und M als Multiply
(Multiplizieren) auf.

\[
    \text{SM\hspace{-17pt}{\textemdash}{--}\hspace{2pt}
        SM SM SM SM S SM S S S}
\] \[
    ((((((((3^2 \cdot 3)^2 \cdot 3)^2 \cdot 3)^2
        \cdot 3)^2)^2 \cdot 3)^2)^2)^2
\]

Insgesamt müssen wir jetzt nur mehr 14 Operationen statt exakt 1000
Operationen durchführen.

\section{Prof. Bauer Ulrich}

\subsection{Definieren Sie den Begriff ''Wirtschaft''}

Unter Wirtschaft verstehen wir jenes Gebiet menschlicher Tätigkeiten,
das der Bedürfnisbefriedigung dient. Die menschlichen Bedürfnisse sind
praktisch unbegrenzt, die zur Bedürfnisbefriedigung geeigneten Mittel
(Güter) stehen dagegen nicht in unbeschränkter Menge zur Verfügung,
sondern sind von Natur aus knapp. Dieser Umstand zwingt die Menschen
zu wirtschaften.

\subsection{Definieren Sie den Begriff ''Wirtschaften''}

Unter Wirtschaften verstehen wir das Disponieren über knappe Güter, die
direkt oder indirekt geeignet sind, menschliche Bedürfnisse in möglichst
großem Maße zu befriedigen.

\subsection{Worum handelt es sich beim ökologischen Prinzip?}

Das ökologische Prinzip lässt sich mengen- und wertmäßig formulieren
und schreibt 3 Prinzipien vor. Mit einem gegebenen Aufwand an
Wirtschaftsgütern soll ein möglichst hoher Ertrag erzielt werden
(Maximumprinzip). Der nötige Aufwand, um einen bestimmten Ertrag zu
erzielen, wird möglichst gering gehalten (Minimumprinzip). Und ein
möglichst günstiges Verhältnis zwischen Aufwand und Ertrag muss
realisiert werden (generelles Extremumprinzip).

\subsection{Was bezeichnet den Begriff des ''Entrepreneurship''?}

\fbox{Vorsicht!} Es handelt sich um keine Definition des Vortragenden.

Das Entrepreneurship ist eine Teildisziplin der Wirtschaft und
befasst sich mit der Neugründung von Unternehmen und dem unternehmerischen
Handeln. Es befasst sich mit der Beobachtung, wann sich Arbeitnehmer
selbstständig machen und Unternehmen gründen. Eine deutsche Übersetzung
wäre ''Unternehmertum''.

\section{Prof. Bischof Horst}

\subsection{Prof. Bischof gab eine Definition des Begriffes ''Informatik''.
    Ich bitte um Ihre Definition.}

\begin{quote}
    Informatik ist die Wissenschaft von der systematischen und
    automatisierten Verarbeitung von Information. Sie erforscht
    grundlegende Verfahrensweisen der
    Informationsverarbeitung und allgemeine Methoden ihrer
    Anwendung in den verschiedenen Bereichen. Für diese
    Aufgaben wendet die Informatik vorwiegend formale und
    ingenieurmäßig orientierte Techniken an. \\
    -- Gesellschaft für Informatik
\end{quote}

Die Informatik ist eine Wissenschaft, die sich mit dem Interpretieren,
Strukturieren und der Verarbeitung von Information als Daten befasst
und Werkzeuge zur automatisierten Bearbeitung bereitstellt.

\subsection{Wofür steht das Akronym CIA aus dem Bereich der
    Informationssicherheit?}

In der Informationssicherheit verwendet man Verschlüsselungsmechanismen
um Vertraulichkeit (confidentiality), Integrität (integrity) und 
Authenzität (authenticity) zu erreichen.

\subsection{Welche LVs aus dem Bereich der Programmierung sind im Studium
    Softwareentwicklung-Wirtschaft enthalten?}

\uncertain

\begin{itemize}
  \item Einführung in die strukturierte Programmierung
  \item Softwareentwicklung Praktikum
  \item Softwareparadigmen
  \item Entwurf und Entwicklung großer Systeme
\end{itemize}

\subsection{Es wurde der Begriff ''ubiquitäres Rechnen'' verwendet. Geben
    Sie eine Definition}

Unter Ubiquitous Computing bezeichnet die Allgegenwärtigkeit der
rechnergestützten Informationsverarbeitung.

In der Vorlesung wurde
die technologische Entwicklung der Geräte seit den 60ern betrachtet.
Während früher große Mainframes von zahlreichen technisch versierten
Personen gemeinsam verwendet wurden, führte der Trend bis (etwa) zur
Jahrtausendwende hin zum Personal Computer; einem Gerät, welches nur
von einer Person verwendet wird. Mit dem Beginn des ubiquitären Rechnens
besitzt eine einzelne Person mehrere Geräte, die für verschiedene Aufgaben
zur Verfügung stehen (als größtes Beispiel sei hier das Handy/Smartphone
genannt, weitere sind Handhelds, Netbooks, Chips). In der Zukunft
erwartet man einen verstärkten Einsatz des Cloud Computing (Verwendung
von Servern zur Datenspeicherung und Datenverarbeitung).

\section{Prof. Wotawa Franz}

\subsection{Welche historische Definition des Begriffs ''Engineering'' im
    Kontext von ''Software Engineering'' kennst du?}

\begin{quote}
    ''To define it rudely but not ineptly,
    \emph{engineering} is the art of doing
    that well with one dollar,
    which any bungler can do with
    two after a fashion.'' \\
    -- Arthur Wellesley, 1769--1852
\end{quote}

\subsection{Es wurden die Inhalte der Lehrveranstaltung
    ''Objektorientierte Analyse und Design'' vorgestellt. Worum geht es?}

In der LV geht es um die Veranschaulichung von objektorientierten
Lösungsansätzen (Analyse \& Design). Teil der Lehrveranstaltung ist die
Auseinandersetzung mit der Modellierungssprache UML (Unified Modeling
Language). Zitat aus der LV-Übersicht:

\begin{itemize}
  \item Grundprinzipien der OO Analyse
  \item Grundprinzipien des OO Designs
  \item Entwicklungsprozess: von der Analyse zur Implementierung von OO
Softwaresystemen
  \item Rational Unified Process
  \item Use Cases
  \item Klassendiagramme (auf Analyse- und Designebene)
  \item Sequenzdiagramme
  \item Kollaborationsdiagramme
  \item Komponentendiagramme
  \item State Charts (inkl. der Abbildung von Sequenzdiagrammen auf State Charts)
  \item Ableitung von Relationenschemata aus UML Klassendiagrammen
  \item Ableitung von Quellcode aus Komponentendiagrammen und State Charts
  \item Entwicklung eines durchgehenden Anwendungsbeispiels (in Teams) 
\end{itemize}

\section{Prof. Aichholzer Oswin}

\subsection{Als Beispiel eines praktisch relevanten Themas, das jedoch
    viel Theorie-Lastigkeit hat, stellte Dozent Aichholzer den
    ''Spannbaum'' dem ''Rundreiseproblem'' gegenüber.
    Worum handelt es sich?}

Beim Spannbaum-Problem verfolgt man das Ziel mehrere Punkte, die auf
einer Fläche verteilt sind, mit Linien zu verbinden, wobei die
Gesamtlänge der Linien minimal gehalten werden soll.

Das Rundreiseproblem geht man ebenso von verteilten Punkte auf einer
Fläche aus und versucht von einem Punkt ausgehend mit einer Linie
alle Punkte zu berühren, wobei der Endpunkt der Anfangspunkt sein
muss. Wieder soll die Gesamtlänge minimal gehalten werden.

Wobei diese beiden Probleme sehr ähnlich aufgebaut sind, so kann man das
Spannbaumproblem zB bei einem Eingabewert von 20 Punkten bereits
408fach schneller lösen als das Rundreiseproblem. In der Praxis verwendet
man deshalb für das Rundreiseprobleme heuristische Algorithmen bzw.
Näherungsverfahren an, um nicht die nachweislich optimale, sondern eine
möglichst optimale unter Berücksichtigung der Wirtschaftlichkeit zu
erreichen. Das Spannbaumproblem ist Teil der Klasse $\mathcal{O}(n\log n)$,
während das Rundreiseproblem NP-vollständig ist.

\end{document}
