\documentclass[a4paper,twocolumn]{article}

\usepackage[utf8]{inputenc}
\usepackage{amssymb}
\usepackage{amsmath}
\usepackage{amsthm}
\usepackage{textcomp}

\usepackage[dvips,pdftex]{geometry}
\usepackage{graphicx}
\usepackage{listings}
\usepackage{pifont}
\DeclareMathOperator{\rank}{rank}
\DeclareMathOperator{\ld}{ld}

\newcommand{\set}[1]{\left\{#1\right\}}
\newcommand{\io}[2]{{\par\noindent\textbf{Input:} #1 \\}{\textbf{Output:} #2 \\}}
\newcommand{\wa}[1]{\textbf{Wolframalpha:} $#1$ \\}
\newcommand{\scriptref}[1]{\textbf{Skriptum.} page #1 \\}
\newcommand{\dt}[1]{\textbf{Deutsch } #1 \\}

\begin{document}
\tableofcontents

\section{Basic Linear Algebra}

\[
    0 := \begin{pmatrix}
        0 & \ldots & 0 \\
        \vdots & \ddots & \vdots \\
        0 & \ldots & 0 \\
    \end{pmatrix}
\] \[
    A = B \Leftrightarrow a_{ij} \in A = b_{ij} \in B
\] \[
    A + 0 = 0 + A = A
    \qquad
    A + (-A) = 0
\] \[
    A \cdot x = b
    \quad\Rightarrow\quad
    x = A^{-1} \cdot b
\] \[
    a_{ij} \in A \Leftrightarrow (-a_{ij}) \in (-A)
\] \[
    A + (B + C) = (A + B) + C
    \qquad
    A(BC) = (AB)C
\] \[
    (\lambda + \mu) \cdot A = \lambda \cdot A + \mu \cdot A
\] \[
    A \text{ has $m$ rows and $n$ columns}
        \Leftrightarrow A \in M(m\times n)
\]

A linear system of equations can be translated to matrizes
directly.
%
\[
    \begin{array}{cccccc}
        -3x & + & -3y & -3z & = & -3 \\
        -2x & + &  2y &   z & = & 0 \\
          x & + & -3y &  3z & = & 0 \\
    \end{array}
\] \[
    \Rightarrow
    A = \left(\begin{array}{cccc}
        -3 & -3 & -3 & -3 \\
        -2 &  2 &  1 & 0 \\
         1 & -3 &  3 & 0 \\
    \end{array}\right)
\]

To create an extended coefficient matrix, you have to add the solutions
as the most right column (thus $\{A, b\}$).
If you want to solve this system, you have to find $x$ in
$A\times x = b$, where $A$ are the coefficients of the system and
$b$ are the solutions.

This can be done by performing arithmetic row-wise operations.
An example is taking the third row minus $\frac12$ times the second
row. The result is a tuple $(0, -4, 2.5, 0)$ which can replace . Each operation returns
a factor (here: $-\frac12$); we will need this for decompositions.
Okay, the result can be taken as new second row. So how is the general
algorithm?

\subsection{Gaussian elimination}
\io{Invertible square matrix}{triangular form}
\wa{\operatorname{RowReduce}[A]}
\scriptref{9}

In a $m\times n$ matrix ($m$ rows, $n$ columns), we want to reach
the structure of an upper triangular matrix. So if the result of such
an operation is $(0, -4, 2.5, 0)$, we will prefer to use it as the
second row (because of the one zero to the left). We will perform operations
until we reach the expected structure; the ''triangular form'' (numerical
analysis) or ''row echelon form'' (abstract algebra).
%
\[
    \left(\begin{array}{cccc}
        -3 & -3 &   -3 & -3 \\
        0  & -4 &  2.5 &  0 \\
        0  &  0 & -0.5 & -1 \\
    \end{array}\right)
\]

This structure can be easily transformed to the solution of the linear
system.
%
\begin{minipage}{0.2\textwidth}\[
    -0.5z = -1
\] \[
    -4y + 2.5 \cdot (-1) = 0
\] \[
    y = \frac54
\]
\end{minipage}\begin{minipage}{0.3\textwidth}
\[
    -3x - \frac{15}{4} = \frac{12}{4}
\] \[
    x = -\frac94
\]\end{minipage}

Otherwise we can continue the elimination algorithm to create an
identity matrix on the left side (columns 1--3 here). This way
we can read the variable values immediately. This algorithm is
called Gauss-Jordan Elimination.

\subsection{Notes}

\begin{itemize}
  \item Pivot elements are the most-left numbers of the rows.
        $-3$, $-4$ and $-0.5$ in the  previous example.
  \item $\rank{A}$ is the number of rows with non-zero pivot elements
        in the triangular matrix of $A$. $\rank{A} = 3$ in the
        previous example.
  \item Sometimes swapping rows is necessary to reach a triangular
        structure.
  \item If there are only non-zero pivots, there is only one solution
        for the linear system. Otherwise we don't know anything about
        solutions.
  \item There are 3 operations that can be performed (elementary row
        operations):
    \begin{itemize}
      \item Swapping rows
      \item multiplication of a row with $\lambda \neq 0$
      \item addition of row i with row j
    \end{itemize}
  \item If there is only one solution the equivalent equation
        ($A\cdot x = 0$) can only be solved by $x = 0$.
  \item The system $Ax = b$ has a solution if 
        \[
            \rank{\left(A\mid b\right)} = \rank{(A)}
        \]
  \item There is one unique solution if
        \[
            \rank{\left(A\mid b\right)} = \rank{(A)} = n
        \]
  \item There is one unique solution in a quadratic system if
        $\det{(A)} \neq 0$.
\end{itemize}

\subsection{Matrix multiplication}

\io{Two matrices $A$ and $B$ where
    number of $\text{columns}(A) =$ number of $\text{rows}(B)$
}{One matrix $C$}
\wa{A * B}
\scriptref{5}
%
\[
    A \cdot B = C
\] \[
    \begin{pmatrix}
        2 & 3 & 6 \\
        4 & 5 & 3 \\
    \end{pmatrix}
    \cdot
    \begin{pmatrix}
        1 & 3 \\
        2 & 4 \\
        7 & 8 \\
    \end{pmatrix}
    =
    \begin{pmatrix}
        50 & 66 \\
        35 & 56 \\
    \end{pmatrix}
\]

$50$ is the sum of $2\cdot1 + 3\cdot2 + 6\cdot7$.
Matrix multiplication is \emph{not} commutative.

\subsection{Arithmetic matrix operations}

Matrix additions or operations with a scalar happen element-wise.

\subsection{Set operations}

See figure~\ref{fig:set_operations}. Copyright Wikipedia.
%
\begin{figure}[h]
  \begin{center}
    \includegraphics[scale=0.5]{set_operations.pdf}
    \caption{
        Basic set operations from left to right:
        first row: Intersection ($\cap$), union ($\cup$).
        second row: relative complement ($A \setminus B$),
            complement $U \setminus A$ (U is universal set),
            symmetric difference ($A \Delta B = (A\setminus B) \cup (B
            \setminus A)$)
    }
    \label{fig:set_operations}
  \end{center}
\end{figure}

\subsection{Transposed matrix}

\io{A matrix $A \in M(m\times n)$}{Matrix $A^T \in M(n\times m)$}
\wa{\operatorname{Transpose}[A]}
\scriptref{6}
%
\[
    \begin{pmatrix}
        12 & 6 \\
        2  & 3 \\
        4  & 8 \\
    \end{pmatrix}^T
    =
    \begin{pmatrix}
        12 & 2 & 4 \\
        6 & 3 & 8 \\
    \end{pmatrix}
\] \[
    (a_{ij})^T = (a_{ji})
\] \[
    (A + B)^T = A^T + B^T
\] \[
    (A\cdot B)^T = B^T\cdot A^T
\] \[
    A = A^T \Leftrightarrow A\text{ is a ''symmetrical matrix''}
\]

\subsection{Inverse matrix}

\io{$A \in M(n\times n), \rank{A} = n$}{$C \in M(n\times n)$ if $A$ is regular}
\wa{A^{-1}}
\scriptref{16}

If $A$ has an inverse matrix, $A$ is called ''regular matrix'';
''singular'' otherwise.
%
\[
  \begin{array}{lrcl}
    A\text{ is regular}: & (A \cdot B)^{-1} &=& B^{-1} \cdot A^{-1} \\
    A^{-1}\text{ is regular}: & (A^{-1})^{-1} &=& A \\
    A^T\text{ is regular}: & (A^T)^{-1} &=& (A^{-1})^T \\
  \end{array}
\]
%
\[
    A \in M(n\times n) \quad \Rightarrow A\cdot A^{-1} = I
\]

The inverse matrix of $A$ is $A^{-1}$. This matrix can be found by
solving the system:
%
\[
    (A, I) = \left( \begin{array}{ccc|cccc}
        a_{11} & \ldots & a_{1n} & 1 & 0 & \ldots & 0 \\
        a_{21} & \ldots & a_{2n} & 0 & 1 & \ldots & 0 \\
        \vdots & \ddots & \vdots & \vdots & \vdots & \ddots & \vdots \\
        a_{n1} & \ldots & a_{nn} & 0 & 0 & \ldots & 1 \\
    \end{array} \right)
\]

Once there is an identity matrix on the left side (by elementary row-wise
operations), the inverse matrix can be read from the right side of the
separator.

\section{Decompositions}

\begin{description}
  \item[LUP decomposition]
    $
        PA = LU
    $
  \item[LDU decomposition]
    $
        A = LDU
    $
  \item[LU decomposition with full pivoting]
    $
        PAQ = LU
    $
\end{description}

\subsection{LU decomposition}

\io{$A \in M(m\times n)$}{$L$ and $U$ with $A=LR$}
\wa{\operatorname{LUDecomposition}[]}
\scriptref{18}
%
\[
    A = L \cdot U
    \quad\text{(without swapped rows)}
\]
%
\begin{itemize}
  \item U is matrix A in triangular form
  \item L is a quadratic matrix with ones in the diagonal and below
        negative factors created by row-wise operations performed before.
\end{itemize}

\[
    A = P^T \cdot L \cdot R
    \quad\text{(with swapped rows)}
\]
%
\begin{itemize}
  \item U is matrix A in triangular form
  \item L is a quadratic matrix with ones in the diagonal and below
        negative factors created by row-wise operations performed before.
        While swapping rows, you have to swap all components of the rows
        left to the right ones accordingly.
  \item Permutation matrix $P$ can be created by constructing an identity
        matrix and swapping all rows you did with L. $P^T = P^{-1}$.
\end{itemize}

A solution for $Ax = b$ can be found by applying
\[
    L\cdot y = b
\] \[
    R\cdot x = y
\]

\subsection{QR decomposition}

\io{$A \in M(m\times n), m \geq n$ and linear independent column vectors}
    {$Q(m\times n)$ and $R(n\times n)$ with $A = QR$}
\wa{\operatorname{QRDecomposition}[A]}
\scriptref{58}
%
\begin{enumerate}
  \item Apply Gram-Schmidt-Process to column vectors
  \item $Q = \{q_1, q_2, \ldots\}$
  \item $Q^T A = R \Rightarrow R$
\end{enumerate}
\[
    a
\]

$Q$ is a matrix of orthonormal column vectors and $R$ is an invertible upper
triangular matrix.

\section{Determinant}

\io{$A \in M(n\times n)$}{a scalar}
\wa{\operatorname{Det}[A]}
\scriptref{21}
%
\[
    n=1: \det{A} = (a,) := a
\] \[
    n=2: \det{A} := \begin{vmatrix}
        a_{11} & a_{12} \\
        a_{21} & a_{22} \\
    \end{vmatrix} :=
    a_{11}\cdot a_{22} - a_{12}\cdot a_{21}
\]

For $n=3$, it's recommended to evaluate the determinant by using
the rule of Sarrus.
This can be done by adding as many column duplicates as necessary to
get valid diagonals (simply $2m - 1$). This structure allows you to simply
read all necessary addition and subtraction operations
(see figure~\ref{fig:sarrus}).
%
\begin{figure}[h]
  \begin{center}
    \includegraphics[width=150pt]{sarrus.pdf}
    \caption{Rule of sarrus}
    \label{fig:sarrus}
  \end{center}
\end{figure}

\[
    \det{A} = 1\cdot 1\cdot 1 + 2\cdot 3\cdot(-1) + (-1)\cdot 0\cdot 2
\] \[
    - [(-1) \cdot 1 \cdot (-1)] - [1\cdot 3\cdot 2] - [2\cdot 0\cdot 1]
    = -12
\]
%
The algebraic complement $A_{ij}'$ of $A$ is the determinant
of the $(n-1)\times(n-1)$ matrix created by removing row $i$
and column $j$.
%
\[
    A = (a_{ij}) \in M(n\times n)
\] \[
    \det{A} = \sum_{j=1}^n (-1)^{1+j} a_{1j} A_{ij}'
\]
%
\begin{itemize}
  \item $\det{A^T} = \det{A}$
  \item Entire row or column is zero $\Rightarrow \det{A} = 0$
  \item $A\in M(n\times n), \lambda \in \mathbb{K}:
        \det{(\lambda A)} = \lambda^n \det{A}$
  \item Two rows or columns are identical: $\det{A} = 0$
  \item $\det{(A\cdot B)} = \det{A}\cdot\det{B}$
  \item $\det{(A + B)} \neq \det{A} + \det{B}$
\end{itemize}
%
\[
    A \text{ is regular}
    \Leftrightarrow
    \rank{A} = n
    \Leftrightarrow
    \det{A} \neq 0
\]

\section{Vectors}

\wa{\{\{1\},\{2\},\{3\}\}}
\scriptref{24}
%
\[
    \lambda, \mu \in \mathbb{R}
\]
%
\begin{itemize}
  \item Vector $\vec{v}$ with $\|\vec{v}\| = 0$ is called Null vector.
  \item Vector $\vec{v}$ with $\|\vec{v}\| = 1$ is called Unit vector.
  \item $\vec{a} + 0 = \vec{a}, \quad \vec{a} \cdot 0 = 0$
  \item $\vec{a} + \vec{b} = \vec{b} + \vec{a}$
  \item $(\vec{a} + \vec{b}) + c = \vec{a} + (\vec{b} + c)$
  \item $\|\lambda\cdot \vec{a}\| = |\lambda| \cdot \|\vec{a}\|$
  \item $\lambda\cdot (\vec{a} + b) = \lambda\cdot \vec{a} + \lambda\cdot b$
  \item $(\lambda + \mu)\cdot \vec{a} = \lambda\cdot \vec{a}
        + \mu\cdot \vec{a}$
  \item $\|\vec{a} + \vec{b}\| \leq \|\vec{a}\| + \|\vec{b}\|$
\end{itemize}

\section{Norm (length) of a vector}
%
\io{A vector $\vec{v}$}{A scalar}
\wa{\operatorname{Norm}[\{2,3\}]}
%
\[
    \|\vec{v}\| = \sqrt{\sum_i v_i^2}
\]

\section{Products of vectors}

\subsection{Dot product (of vectors)}

\io{2 arbitrary vectors $\vec{a}$ and $\vec{b}$}{A scalar}
\wa{\operatorname{vector}\{1,2,3\} . \operatorname{vector}\{2,3,4\}}
\scriptref{26}
\dt{''Skalarprodukt''}
%
\[
    \langle a, b\rangle = \|a\|\cdot\|b\| \cdot \cos{\varphi}
\]
%
\begin{itemize}
  \item $\langle \vec{a}, \vec{b}\rangle = \langle \vec{b}, \vec{a}\rangle$
  \item $\langle \vec{a}, \vec{b}+\vec{c}\rangle = \langle \vec{b},
        \vec{a}\rangle + \langle \vec{a},\vec{c}\rangle$
  \item $\langle \vec{a}, \vec{b}\rangle = 0 \Leftrightarrow
        \vec{a}\bot \vec{b}$
  \item $\langle \vec{a}, \vec{a}\rangle = \|\vec{a}\|^2$
  \item $\vec{a} \in \mathbb{R}^3:
        \langle \vec{a},\vec{b} \rangle = a_1b_1 + a_2b_2 + a_3b_3$
  \item $\vec{a} \in \mathbb{R}^3:
        \|\vec{a}\| = \sqrt{\langle \vec{a}, \vec{a}\rangle}
        = \sqrt{a_1^2 + a_2^2 + a_3^3}$
\end{itemize}

\subsection{Cross product}

\io{2 arbitrary vectors $\vec{a}$ and $\vec{b}$}{A scalar $c$}
\wa{\{1,2,3\} \operatorname{cross} \{2,3,4\}}
\scriptref{27}
\dt{''Vektorprodukt''}
%
\[
    \vec{a}, \vec{b}\in\mathbb{R}^3:
        \quad c := \vec{a}\times \vec{b}
\] \[
    c := \|\vec{a}\| \cdot \|\vec{b}\| \cdot \sin{\varphi}
\]
%
Example:
%
\[
    \begin{pmatrix} 2 \\ 4 \\ 5 \end{pmatrix}
        \times
    \begin{pmatrix} 3 \\ 6 \\ 1 \end{pmatrix}
    =
    \begin{pmatrix}
        4\cdot1 - 5\cdot6 \\
        -(2\cdot1 - 5\cdot3) \\
        2\cdot6 - 4\cdot3 \\
    \end{pmatrix}
    =
    \begin{pmatrix}
        -26 \\
        13 \\
        0 \\
    \end{pmatrix}
\]

\subsection{Triple product}

\io{3 arbitrary vectors $\vec{a}$, $\vec{b}$ and $\vec{c}$}{A scalar}
\wa{\{1,2,3\} \operatorname{cross} \{2,3,4\}}
\scriptref{27}
\dt{''Spatprodukt''}
%
\[
    \vec{a}, \vec{b}, \vec{c} \in \mathbb{R}^n
\] \[
    |\langle \vec{a}\times \vec{b}, \vec{c}\rangle|
        = \|\vec{a}\times \vec{b}\| \cdot \|\vec{c}\| \cdot
        \cos{\angle(\vec{a}\times \vec{b}, \vec{c})}
\]

\section{Linear maps}

\begin{itemize}
  \item injective (injections can be undone)
        \[
            \forall x_1, x_2 \in A: f(x_1) = f(x_2)
                \Leftrightarrow x_1 = x_2
        \]
  \item surjective (each element has a root):
        \[
            \forall y \in B: \exists x \in A:
                f(x) = y
        \]
  \item bijective = injective and surjective
\end{itemize}

\begin{figure}[h]
  \begin{center}
    \includegraphics[scale=0.5]{set_relations.pdf}
    \caption{Relations of sets (top to bottom)
        1. injective
        2. surjective
        3. bijective
    }
    \label{fig:set_relations}
  \end{center}
\end{figure}

\section{Vector spaces}

$\mathbb{K}$ is either $\mathbb{R}$ or $\mathbb{C}$.
$\mathbb{P}_m$ is vector space of polynomials with degree
$m$ at maximum.
A non-empty set $V$ is called vector space of $\mathbb{K}$ if

\begin{enumerate}
  \item The sum of $a, b \in V$ is defined and in $V$
  \item The product of $\lambda \in \mathbb{K}$ and $a \in V$
        ($= \lambda \cdot a$) is defined and an element of $V$
\end{enumerate}

\begin{itemize}
  \item null vector: $0 \in V, \forall a \in V: a + 0 = a$
  \item unit vector: $1 \in V, \forall a \in V: a \cdot 1 = a$
  \item negative vector: $(-a) \in V, a + (-a) = 0$
  \item Is $V$ a vector space of $K$. A non-empty subspace $U \subset V$
        is called ''linear subspace of $V$'' if $\lambda \in K$ and
        $a, b \in U$ with
        \[
            a + b \in U,
                \quad \lambda \cdot a \in U
        \]
  \item A line in $\mathbb{R}^2$ and $\mathbb{R}^3$ containing the origin
        are linear subspaces.
  \item A plane in $\mathbb{R}^3$ containing the origin is a linear
        subspace of $\mathbb{R}^3$.
\end{itemize}

\section{Linear independency}

A non-empty subset $U \subset V$ is called \emph{linear independent} if
a finite number of vectors in $U$ are linear independent.
%
\[
    \lambda_1, \ldots, \lambda_m \in K,
    \quad a_1, \ldots, a_m \in V
\]

A vector like
%
\[
    a = \lambda_1 a_1 + \ldots + \lambda_m a_m
\]
%
is called linear combination of vectors ($a_1,\ldots,a_m$). A linear
combination is \emph{trivial} if $\lambda_1 = \ldots = \lambda_m = 0$.

Vectors $a_1,\ldots,a_m$ are linear dependent if there is a non-trivial
linear combination with
%
\[
    \lambda_1 a_1 + \ldots + \lambda_m a_m = 0
\]
%
These vectors are linear independent if 
%
\[
    \lambda_1 = 0, \ldots, \lambda_m = 0
\]

Is $U \subset V$ is a non-empty subspace. The set of all linear combinations
of vectors of $U$ ($= L(U)$) is called the spanned space by $a_i \in U$.

\[
    L(U) = \left\{ a = \sum_i \lambda_i a_i,
            \lambda_i \in K, a_i \in U\right\}
\]

Is $U = \{a_1,\ldots,a_m\}$:

\[
    L(U) = L(a_1, \ldots, a_m)
\]

\section{Base of a vector space}

$V$ is a vector space of $K$. A subspace $U \subset V$ of linear
independent vectors is called basis of $V$ if $L(U) = V$.
A vector space with a finite basis is called finite-dimensional.

All bases in a finite-dimensional vector space $V$ have the same number
of vectors. This number is called dimension ($\dim{V}$).

\begin{itemize}
  \item The base of $\mathbb{R}^n$ is $\set{e_1,\ldots,e_n}$.
        $\dim{\mathbb{R}^n} = n$
  \item The base of $\mathbb{P}^m$ is $\set{1, x,\ldots,x^m}$.
        $\dim{\mathbb{P}_m} = m + 1$
\end{itemize}

$V$ is a vector space with basis $B = \set{v_1,\ldots,v_n}$. Each vector
$v \in V$ can be created by unique scalars $\lambda_1,\ldots,\lambda_n$.
%
\[
    v = \lambda_1 v_1 + \ldots + \lambda_n v_n
\]

\section{Diagonalisation}

\io{A diagonalisable matrix $A \in M(n\times n)$}
   {diagonal matrix $D$ with $B = C^{-1}DC$}
\wa{\operatorname{Diagonalize}[A]}
\scriptref{63}

The eigenvalues of a diagonal matrix are the diagonal elements.
$A$ is diagonalisable if $A$ has $n$ linear independent eigenvectors.
$C$ can be created by linear independent eigenvectors of $A$ as
columns in $C$. This structure satisfies $D = C^{-1} AC$.

\section{Eigenvalues}

\io{$A \in M(n\times n)$}{a scalar $\lambda$}
\wa{eigenvalues[A]}
\scriptref{59}
%
\[
    \det{(A - \lambda I)} = 0
\]
with $\lambda$ as unknown variable. If you resolve the determinant,
you will get a polynomial you want to know the Zero of. This polynomial
is called ''characteristic polynomial of $A$''. In a polynomial of
degree $n$ you will get $n$ solutions and therefore $n$ eigenvalues
$\lambda$.
Row swapping is allowed.

$\lambda$ satisfies:
\[
    A\cdot v = \lambda\cdot v
\]

\section{Eigenvectors}

\io{$A \in M(n\times n)$ and eigenvalues $\sigma_{1,\ldots,n}$}
   {$n$ linear independent vectors}
\wa{\operatorname{eigenvectors}[A]}
\scriptref{59}

If $\sigma_{i} \neq \sigma_j \quad\forall i, j \in [1,n], i \neq j$
then linear independence is given for eigenvectors for sure.

\[
    (A - \lambda_i I) \cdot v_i = 0
\]
Solve this equation system for $v_i$ which will be your eigenvector.

\section{Gram-Schmidt process}

\io{2 or more vectors}{as many vectors as given by input}
\wa{\operatorname{Orthogonalize}[\{A,B,C\}]}
\scriptref{56}
%
\[
    w_1 = \frac{1}{\| v_1\|}\cdot v_1
\] \[
    w_i = \frac{1}{\| u_i\|}\cdot u_i, \quad i = 2,\ldots,n
\] \[
    u_i = v_i - \sum_{k=1}^{i-1} \langle v_i, w_k\rangle w_k, \quad i = 2,\ldots,n
\]

\section{Pseudoinverse}

\io{Matrix $A \in M(m\times n)$}{$A^{\#} \in M(n\times m)$}
\wa{\operatorname{PseudoInverse}[A]}
\scriptref{92}
\textbf{More precise name:} Moore-Penrose-Inverse

\begin{enumerate}
    \item Evaluate $A^T\cdot A$
    \item Evaluate $(A^T\cdot A)^{-1}$
    \item Evaluate $(A^T\cdot A)^{-1} \cdot A^T = A^\#$
\end{enumerate}
%
Probably the evaluation of the inverse is impossible. In this case,
the pseudoinverse might be possible to evaluate using the singular
value decomposition (SVD):
\[
    A = U\Sigma V^T
        \Rightarrow A^\# = V\Sigma^\# U^T
\]
%
$\Sigma^\#$ can be created by inverting all singular values
in D: $\sigma_1^{-1}$, $\sigma_2^{-1}$, $\sigma_i^{-1}$.

\section{Singular value decomposition}

\io{Matrix $A \in M(n\times n)$}{Matrices $U$, $\Sigma$, $V$
    where $A = U\Sigma V^T$}
\wa{\operatorname{SVD}[A]}
\scriptref{68}
\dt{Singulärwertzerlegung}
%
\[
    A(m\times n) = U(m\times m) \cdot \Sigma(m\times n)
        \cdot V(n\times n)^T
\]
If $A$ is positiv definit and symmetrical, the procedure is the same like
orthogonal diagonalisation.
%
\begin{enumerate}
    \item Evaluate $A^T\cdot A$
    \item Evaluate die eigenvalues of
          $A^T\cdot A: \lambda_1, \lambda_2, \ldots$
    \item Sort the eigenvalues by value
    \item The singular values $\sigma_1, \sigma_2, \ldots$
          are the squareroots of the eigenvalues $\sqrt{\lambda_1},
          \sqrt{\lambda_2}, \ldots$
    \item Evaluate the eigenvectors $v_1, v_2, \ldots$
    \item Normalize the eigenvectors ($\leftarrow$ length is 1)
    \item Combine the eigenvectors as column vectors $\{v_1, v_2, \ldots\} = V$
    \item Create a $n\times n$ matrix and insert the $\sigma_i$ as
          diagonals:
          \[
            \Sigma = 
            \begin{pmatrix}
                \sigma_1 & 0 & 0 \\
                0 & \sigma_2 & 0 \\
                0 & \vdots   & 0 \\
                0 & 0 & \sigma_n \\
            \end{pmatrix}
          \]
    \item Evaluate $u_i = \frac{1}{\sigma_i} \cdot A \cdot v_i$ or find
          other orthonormal vectors
    \item Combine the vectors as column vectors $\{u_1, u_2, \ldots\} = U$
    \item Evaluate $V^T$
\end{enumerate}
%
$U$ and $V$ are not unique.

\section{Gauss-Seidel Iteration}

\io{Linear equation system $(A\mid b)$ and start vector $x_0$}
   {A vector close to the solution of the equation system}
\scriptref{87}

If no start vector is given, $\begin{pmatrix} 1 \\ 0 \\ 1 \end{pmatrix}$ is
prefered.

The Gauss-Seidel-Iteration converges if
\begin{itemize}
  \item either $A$ is positive definit
  \item or for each eigenvector $\lambda$ of $S^{-1} T$
        it states $|\lambda| < 1$
\end{itemize}

\section{Spectral radius}

\io{$A$ of an iteration algorithm}{a scalar}
\scriptref{86}

In the context of iteration algorithms, $A$ is typically
$S^{-1}T$.
%
\[
    \rho(A) := \max_i{|\lambda_i|}
\]
%
with $\lambda_i$ as eigenvalue of $A$.

\section{Condition number}

\io{regular matrix $A \in M(n\times n)$}{a scalar}
%\wa{N/A}
\scriptref{82}

\[
    \operatorname{cond}(A) := \|A^{-1}\| \cdot \|A\|
\] \[
    \|A\|_\infty = \max{\left\{
        \sum_{j=1}^n |a_{ij}|  \mid i = 1,\ldots,n
    \right\}}
\]

\section{Interpolation and Approximation}

\begin{figure}[h]
  \begin{center}
    \includegraphics[scale=0.4]{interpolation.png}
    \caption{Interpolation}
    \label{fig:interpolation}
  \end{center}
\end{figure}
\begin{figure}[h]
  \begin{center}
    \includegraphics[scale=0.4]{approximation.png}
    \caption{Approximation}
    \label{fig:approximation}
  \end{center}
\end{figure}

\section{Cubic Spline Interpolation}

\io{List of $\{x,y\}$ pairs}{A polynomial}
\wa{\operatorname{BSplineCurve}[pts, SplineDegree\rightarrow3]}
\scriptref{115}

\textbf{Is S(x) a cubic spline interpolation?} \\
\begin{itemize}
  \item The function has to be continous. So all functions must stop
        at the same value at the borders.
  \item $f_i'(x_i) = f_{i+1}'(x_i)$ must be satisfied for all $i$
\end{itemize}

\section{Glossary}

\begin{description}
  \item[identity matrix] $a_{ii} := 1, 0$ otherwise
  \item[positiv definit] $A$ is positiv definit if \emph{one} of
    the following requirements is satisfied:
    \begin{equation}
        x^T A x > 0 \quad \forall x \in \mathbb{R}^n, x\neq 0
    \end{equation} \begin{equation}
        \lambda_i > 0 \quad \forall \lambda_i \text{ of } A
    \end{equation} \begin{equation}
        m > 0 \quad \forall m \text{ as minor of } A
    \end{equation} \begin{equation}
        d_i > 0 \quad \forall d \in A_t
    \end{equation}
    with $d$ as the pivot elements of the triangular form
    (\emph{without} row swapping) of $A$.
  \item[regular matrix] $M(m\times n)$ has an inverse matrix
  \item[singular matrix] $M(m\times n)$ has no inverse matrix
  \item[similar matrix] $A, B \in M(n\times n)$ are similar
        if $C \in M(n\times n)$ in $B = C^{-1} AC$ exists;
        $A$ and $B$ have the same characteristic polynomial and
        the same eigenvalues
  \item[spectral radius] $\rho(S^{-1}T) := \max_i{|\lambda_i|}$
  \item[strongly diagonal dominant] \hfill{}
    \[
        |a_{ii}| > \sum_{\substack{j=1 \\ j\neq i}}^n
        |a_{ij}| \quad \forall i = 1,\ldots,n
    \]
  \item[symmetrical matrix]
    $a_{ij} = a_{ji} \quad\forall a \in M(m\times n)$
  \item[permutation] An identity matrix where the same
    elementary row operations of the Gaussian algorithm
    have been applied on
  \item[triangular form]
    %elements below or above diagonal line are all 0
    $a_{ij} = 0 \quad\forall i > j$ or $\forall i < j$
  \item[quadratic matrix] $A \in M(n\times m): n = m$
\end{description}

\end{document}
