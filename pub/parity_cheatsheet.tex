% parity_cheatsheet.tex
%
% Author:  Lukas Prokop <admin@lukas-prokop.at>
% Date:    10.10.31
% Version: alpha
% License: Emailware
%
% Copyright 2010, Lukas Prokop

\documentclass[10pt]{article}

% PACKAGES
\usepackage[utf8]{inputenc}
\usepackage{multicol}
\usepackage[sf]{titlesec}
\usepackage[dvips,pdftex]{geometry}
\usepackage{amssymb}
\usepackage{amsmath}
\usepackage{amsthm}
\usepackage{graphicx}
\usepackage{pstricks}
\usepackage{pst-node}
\usepackage{pst-plot}
\usepackage{boxedminipage}
\usepackage{booktabs}
\usepackage{listings}
\usepackage[unicode]{hyperref}
\usepackage{textcomp}
\usepackage{array}
\usepackage{fullpage}

% Document Configuration
\thispagestyle{empty}
\pagenumbering{arabic} % Alph, alph, arabic, Roman, roman
\parindent0mm
\parskip2mm
\headheight10mm
\headsep10mm
\setlength{\unitlength}{1cm}
\renewcommand{\thefootnote}{\roman{footnote}}
\renewcommand{\theequation}{\alph{equation}}

% Aliases
% - general
\newcommand{\TODO}[1]{}%\fbox{\color{blue} TODO} #1\par}
\newcommand{\note}[1]{\fbox{\textbf{Note:}} #1\par}

% - math
% --- functions
%\newcommand{\mod}[1]{\bmod{#1}}
\newcommand{\eul}[1]{{\varphi}(#1)}
\newcommand{\vect}[2]{ \left ( \begin{array}{c} #1 \\ #2 \end{array} \right ) }
\newcommand{\ggt}[2]{\textrm{ggT}(#1, #2)}
\newcommand{\divides}{\hskip1pt\mid\hskip5pt}
\newcommand{\ggT}[2]{\text{ggT}(#1, #2)}
%\renewcommand{\binom}[2]{\begin{pmatrix} #1 \\ #2 \end{pmatrix}}
\newcommand{\maximum}[2]{\textrm{max}(#1, #2)}
\newcommand{\minimum}[2]{\textrm{min}(#1, #2)}

% --- boolean
\renewcommand{\t}{\texttt{\text{T}}\hspace{1pt}}
\newcommand{\nomatht}{\hspace{1pt}T\hspace{1pt}}
\newcommand{\f}{\texttt{\text{F}}\hspace{1pt}}
\newcommand{\nomathf}{\hspace{3pt}F\hspace{1pt}}
\newcommand{\ra}{\rightarrow}
\newcommand{\la}{\leftarrow}
\newcommand{\lra}{\leftrightarrow}
\newcommand{\por}{|}
\newcommand{\pand}{\&}

\newcommand{\truthtable}[5]{
    \begin{table}[!th]
      \begin{tabular}{c|c|c}
        A      & B      & #1 \\
      \hline
        \t  & \t  & #2 \\
        \t  & \f & #3 \\
        \f & \t  & #4 \\
        \f & \f & #5 \\
      \hline
      \end{tabular}
    \end{table}
}

% - pattern.tex related
% --- general

% --- VO stuff
\newtheorem{mathdef}{Definition}
\newcommand{\quoded}{\textit{--- quod erat demonstrandum. ---}}

% --- design, layout
\renewcommand{\epsilon}{\varepsilon}

\author{Lukas Prokop}
\title{Parity cheatsheet}

\begin{document}
\maketitle

\begin{center}
  \begin{boxedminipage}{0.5\textwidth}
    \begin{description}
      \item[n] $n \in \mathbb{N}$
      \item[a,b] even integers
      \item[c,d] odd integers
      \item[I] set of even integers
      \item[K] set of odd integers
      \item[f] any function
      \item[g,h] even functions; satisfying $\big( f(x) = f(-x) \big)$
      \item[o,p] odd functions; satisfying $\big( f(x) \neq f(-x) \big)$
      \item[M] set of even functions
      \item[N] set of odd functions
    \end{description}
  \end{boxedminipage}
\end{center}

\begin{multicols}{2}

\subsection{Addition, Subtraction}

\[
    even \pm even = even
\] \[
    even \pm odd = odd
\] \[
    odd \pm odd = even
\]

\subsection{Multiplication}

\[
    even \times even = even
\] \[
    odd \times odd = odd
\]

\subsection{Facts}


\begin{itemize}
  \item $O(x) = 0$ is the only function $ \in M, N $.
  \item $(O \neq g \land O \neq h) : (g + h) \notin \Big\{ M, N \Big\} $
  \item $(g + h) \in M$
  \item $(g \cdot n) \in M$
  \item $(o + p) \in N$
  \item $(o \cdot n) \in N$
  \item $(g \cdot h) \in M$
  \item $(o \cdot p) \in M$
  \item $\frac{g}{h} \in M$
  \item $\frac{o}{p} \in M$
  \item $\frac{g}{o} \in N$
  \item $g' \in N$
  \item $o' \in M$
  \item $(g \circ h) \in M$
  \item $(o \circ p) \in N$
  \item $(g \circ o) \in M$
  \item $(f \circ g) \in M$
  \item $\int_{-A}^{+A} o = 0$
  \item $\int_{-A}^{+A} g = 2 \cdot \int_{0}^{+A} g$
\end{itemize}

\end{multicols}
\end{document}
