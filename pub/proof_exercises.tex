% proof_exercises.tex
%
% Version: 1.0
%
% Copyright 2010, Lukas Prokop

\documentclass[11pt,a4paper,twocolumn]{article}

% PACKAGES
\usepackage[utf8]{inputenc}

\usepackage{amssymb}
\usepackage{amsmath}
\usepackage{amsthm}
\usepackage{textcomp}
\usepackage{wasysym} % diameter

\usepackage[sf]{titlesec}
\usepackage{fullpage}
\usepackage{booktabs}
\usepackage{fancyhdr}

\usepackage[dvips,pdftex]{geometry}
\usepackage{graphicx}
\usepackage{pstricks}
\usepackage{pst-node}
\usepackage{pst-plot}

\usepackage[unicode,pdfborder={0 0 0}]{hyperref}
\usepackage{boxedminipage}
\usepackage{lastpage}
\usepackage{multicol}
\usepackage{units}
\usepackage{eurosym}
\usepackage{ifthen}
\usepackage{xifthen}
\usepackage{listings}

% Document Configuration
\pagenumbering{arabic}
\pagestyle{plain}
\setcounter{tocdepth}{2}
\parindent0mm
\parskip2mm
\setlength{\unitlength}{1cm}
\renewcommand{\thefootnote}{\arabic{footnote}} % not happy with other ones :-/
\renewcommand{\theequation}{\alph{equation}}

% Aliases
% - general
\newcommand{\mathsymspace}[1]{{\hskip3pt}#1{\hskip3pt}}
\newcommand{\cmd}[1]{{\tt #1}}
\newcommand{\life}[2]{* #1 $\dagger$ #2}
\newcommand{\TODO}{\fbox{\clock \color{blue} TODO}}
\newcommand{\bs}{\textbackslash}

% - math
% --- space
\newcommand{\mathspace}{\hspace{20pt}} % n=1; \mathspace \frac{1}{n}

% --- functions
%\newcommand{\mod}[1]{\bmod{#1}}
\newcommand{\eul}[1]{{\varphi}(#1)}
\newcommand{\vect}[2]{ \left ( \begin{array}{c} #1 \\ #2 \end{array} \right ) }
\newcommand{\ggt}[2]{\textrm{ggT}(#1, #2)}
\newcommand{\divides}{\hskip1pt\mid\hskip5pt}
\newcommand{\ggT}[2]{\text{ggT}(#1, #2)}
%\renewcommand{\binom}[2]{\begin{pmatrix} #1 \\ #2 \end{pmatrix}}
\newcommand{\maximum}[2]{\textrm{max}(#1, #2)}
\newcommand{\minimum}[2]{\textrm{min}(#1, #2)}
\newcommand{\set}[1]{\Big{#1\Big}}

% --- symbols
\newcommand{\ra}{\rightarrow}
\newcommand{\Ra}{\Rightarrow}
\newcommand{\la}{\leftarrow}
\newcommand{\La}{\Leftarrow}
\newcommand{\lra}{\Leftrightarrow}
\newcommand{\average}{\diameter}
\renewcommand{\phi}{\varphi}

% --- complex numbers
\newcommand{\comp}[1]{\overline{#1}}
\newcommand{\conj}[1]{#1 *}

% --- boolean
\newcommand{\andsym}{\mathsymspace{$ \land$}}
\newcommand{\orsym} {\mathsymspace{$ \lor $}}
\newcommand{\notsym}{\mathsymspace{$ \neg $}}
\newcommand{\true}{\hspace{1pt}\text{W}\hspace{1pt}}
\newcommand{\nomathtrue}{\hspace{1pt}W\hspace{1pt}}
\newcommand{\false}{\hspace{3pt}\text{F}\hspace{3pt}}
\newcommand{\nomathfalse}{\hspace{3pt}F\hspace{3pt}}

% - pattern.tex related
% --- general
\newcommand{\linerule}{\hrule{\textwidth}}
\newtheoremstyle{area}
    {6pt}{6pt}{\normalfont}
    {0pt}{\bfseries}{.\hspace{10pt}}{ }{#1}
\theoremstyle{area}
%\theoremstyle{plain}

% --- VO stuff
\newtheorem{mathdef}{Definition}
\newcommand{\question}[1]{\textbf{Q: }#1 \par}
\newcommand{\answer}[1]{\textbf{A: }#1 \par}
\newcommand{\quoded}{
    \parbox{\textwidth}{
        \centering{
            \textsc{
                \small{
                    --- quod erat demonstrandum. ---
                }
            }
        }
    }
}

% --- design, layout
\renewcommand{\epsilon}{\varepsilon}

\title{Exercise collection for mathematical proofs}
\author{Lukas Prokop}
\date{\today}

\begin{document}
\maketitle

\section{Complete Induction}

Show by strong induction for all $n \in \mathbb{N}$:

\subsection{Equations}

\begin{equation}
    \sum_{k=1}^n k = \frac{n (n-1)}{2}
\end{equation}

\begin{equation}
    \sum_{k=0}^n q^k = \frac{q^n - 1}{q - 1}
\end{equation}

\begin{equation}
    \sum_{k=1}^n k (k-1) = \frac13 n (n^2 - 1)
\end{equation}

\begin{equation}
    \sum_{k=1}^n \frac{1}{4k^2 - 1} = \frac12 \Big(1 - \frac{1}{2n + 1}\Big)
\end{equation}

For $a_k \in \mathbb{N}$:

\begin{equation}
    \sum_{k=1}^n a_k \cdot \sum_{k=1}^n \frac{1}{a_k} \geq n^2
\end{equation}

\begin{equation}
    \sum_{l=0}^n \binom{n}{l} = 2^n
\end{equation}

\begin{equation}
    \sum_{l=0}^n (-1)^l \binom{n}{l} = 0
\end{equation}

\begin{equation}
    \sum_{l=0}^n l \binom{n}{l} = n\cdot 2^{n-1}
\end{equation}

Binomial theorem:

\begin{equation}
    (x + y)^n = \sum_{k=0}^n \binom{n}{k} x^k y^{n-k}
\end{equation}

\subsection{Inequations}

\begin{equation}
    2^n > n
\end{equation}

\begin{equation}
    \sum_{k=1}^n (k-1)^2 < \frac{n^3}{3}
\end{equation}

For $n \geq 9$:

\begin{equation}
    2^{2n} \leq n!
\end{equation}

For $n \geq 4$:

\begin{equation}
    3^n > n^3
\end{equation}

For $n \geq 5$:

\begin{equation}
    2^n > n^2
\end{equation}

\subsection{Recursive sequence}

Proof the following statement\dots

\begin{equation}
    a_n = 2 + \frac{1}{2^{n+1} - 1}
\end{equation}

\dots for the following recursive defined sequence:

\begin{equation}
    a_0 = 3; a_n = 3 - \frac{2}{a_n - 1}
\end{equation}

\section{Proof by contradiction}

Meaning: $\sqrt{2}$ is irrational and not rational.
\begin{equation}
    \sqrt{2} \in \mathbb{I} \ni \mathbb{Q}
\end{equation}

Meaning: $\sqrt{5}$ is irrational and not rational.
\begin{equation}
    \sqrt{5} \in \mathbb{I} \ni \mathbb{Q}
\end{equation}

\begin{equation}
    \text{The number of prime numbers is infinite}
\end{equation}

\end{document}
