%\input{../../pattern.tex}
\documentclass[a4paper,twocolumn]{article}
\usepackage[utf8]{inputenc}

\newcommand{\ra}{\rightarrow}
\newcommand{\Ra}{\Rightarrow}

\title{Rechnernetze und -organisation}
\date{11.10.26}
\author{Lukas Prokop}
%{Prof. Posch Karl-Christian}
\begin{document}
\maketitle
\tableofcontents

\section{Terminology}

\begin{description}
  \item[word] fixed size sequence of bits defining a unit.
        Basically word defines the amount of bits to be processed
        at one time in an operation.
  \item[compilation] translation of a program from source language
        to target language (eg. C to machine code)
  \item[32bit system] Machine runs with memory addresses consisting
        of 32 bits
  \item[x86] Well-known family of Intel instruction set architectures
        which is deprecated nowadays in favor of amd64 / x86\_64.
  \item[stack] data structure implementing the LIFO (last in, first out)
        principle
\end{description}

\section{GNU Compiler Collection}

\subsection{Compilation order}

\begin{enumerate}
  \item Write
  \item Compile
  \item Assemble
  \item Link
  \item Load
  \item Execute
\end{enumerate}

\subsection{Procedure results}

\begin{enumerate}
  \item[1.] source code   $\ra$ C
  \item[2.] preprocessing $\ra$ C
  \item[3.] compilated    $\ra$ ASM
  \item[4.] assembled     $\ra$ object file
  \item[5.] linked        $\ra$ executable
\end{enumerate}

\subsection{GCC options}

\begin{description}
  \item[-E] create preprocessed code (step 1 and 2)
  \item[-S] create assembler code.s (step 1, 2 and 3)
  \item[-c] create object file.o (step 1, 2, 3 and 4)
  \item[-o] specify output filename explicitly (not a.out)
  \item[-I] Add the directory dir to the list of directories to be
            searched for header files.
  \item[-Wall] Turn on all optional warnings
  \item[-x] specify programming language explicitly
  \item[-std=gnu89] specify C standard to use
  \item[-O{$[$1230s$]$}] Optimization modes
  \item[-static] Static linking
  \item[-dynamic] Dynamic linking
  \item[-m32] use 32bit architecture instead of current one
  \item[-mbig-endian] Generate code for a big endian target (IA-64 only)
\end{description}

See also ld, cpp, objdump, gdb, as, ar, ranlib and readelf.

\section{x86}

\subsection{Stack subroutine call}

Stack runs against address 0 (in following figure: top is 0).

\begin{table}[ht!]
  \begin{center}
    \begin{tabular}{| c |}
      (ESP-1$\rightarrow$) local var 2 \\
    \hline
      local var 1 \\
    \hline
      (EBP$\rightarrow$) old EBP \\
    \hline
      old EIP \\
    \hline
      arg1 \\
    \hline
      arg2 \\
    \hline
      arg3 \\
    \end{tabular}
  \end{center}
  \caption{Stack dump before subroutine call}
\end{table}

\subsubsection{Call (\texttt{call})}

\begin{enumerate}
  \item store current EIP/PC on stack (ESP)
  \item store old EBP on stack
  \item allocate bytes for local variables
  \item Let EIP/PC be pointer to subroutine
  \item EBP is pointing to old EBP
\end{enumerate}

\subsubsection{Return (\texttt{ret})}

\begin{enumerate}
  \item EBP = old EBP
  \item EIP/PC = return address (old EIP/PC)
  \item ''remove'' function arguments from stack \\
        ESP = ESP + args
\end{enumerate}

\subsubsection{In words}

All storage will be done on a new ''partition (frame)'' on top of the stack.
Therefore save the old EBP on top of stack. Let ESP be the new EBP (EBP =
ESP). Subtract as many memory address from ESP to allocate memory for local
variables. ESP has to keep the \textbf{top} of the stack.

\subsubsection{Callee vs Caller cleanup}

Callee cleanup

\begin{itemize}
  \item + probably more space efficient
  \item - no variadic functions
\end{itemize}

\noindent
Caller cleanup

\begin{itemize}
  \item + variadic functions
  \item + default for x86 C compilers
\end{itemize}

\subsection{x86 registers}

\subsubsection{List}

8 general purpose registers:

\begin{description}
  \item[EAX] arithmetic results data
  \item[EBX] (mov) pointer to data in data segment
  \item[ECX] counter register for string operations
  \item[EDX] port address for I/O -- extending EAX for I/O
  \item[ESI] source index
  \item[EDI] destination index
  \item[EBP] base pointer
  \item[ESP] stack pointer
\end{description}

2 special registers:

\begin{description}
  \item[EIP] instruction pointer, program counter (PC)
  \item[EFLAGS] results from comparisons/tests (implicit usage)
\end{description}

\subsubsection{Description}

\begin{description}
  \item[ESP (Stack pointer)]
        push \& pop \\
        points to top element of stack
  \item[EBP (Base/Frame pointer)]
        references memory of current frame you are in \\
        has to be manipulated explicitly
  \item[EIP (Instruction pointer)]
        holds address of next instruction \\
        Manipulated by jumps and call instructions
\end{description}

\subsection{x86 addressing modes}

\begin{description}
  \item[indexed] In ''x(e)'' you take the register behind x and jump as much
                 bytes forward as register e is telling you
  \item[based] Like indexed, but x is a constant
  \item[immediate] Use parameter value of instruction directly / immediately
\end{description}

In direct mode a register is directly addressed (with its name). In indirect
mode the content of a register is read (by surrounding parentheses).

\subsection{ASM syntax}

x86 is written in Intel syntax (Windows platform) or AT\&T syntax
(UNIX/Linux). We use the second. The following is a short cheatsheet:

\begin{description}
  \item[op C(\%A, \%B, D)]  op *(\%A + C + (\%B * D))
  \item[op C(\%A, \%B)]     op *(\%A + C + \%B)
  \item[op C(\%A), \%B]     op \%A + C, \%B
  \item[op C(,\%A,D), \%B]   op C + \%A * D, \%B
\end{description}

\subsection{Suffixes}

\begin{description}
  \item[q] quad-word  (64bit)
  \item[l] long       (32bit)
  \item[w] word       (16bit)
  \item[b] byte       (8bit)
\end{description}

\subsection{Assembly directives}

\begin{description}
  \item[.string]      Allocate space for a global null-terminated string
  \item[.space]       Allocate bytes for a global
  \item[.rept / endr] Make allocations between .rept and .endr
                      multiple times (according to the parameter)
  \item[.long]        Allocate space for a global long
  \item[.equ]         Define equivalence (two names referring to the same)
  \item[.data]        Data section following. Defines global allocations.
  \item[.file]        Define metadata filename.
  \item[\_start]      Label (for UNIX linker) of program entry point
  \item[.globl]       declare label as global
  \item[.text]        Text section following. Contains program logic.
  \item[.byte]        Allocate a single byte as a global
\end{description}

\subsection{GNU assembler}

\begin{description}
  \item[-a] Write to stdout
  \item[{-}-gstabs] include debugging information to output file.
                  Can eg. be used as additional information for gdb.
\end{description}

Undefined symbols include references to other routines / functions
which will be bound later on by the linker. Defined symbols are labels
and routines already known to the compiler.

\section{Sample boot program}

\begin{enumerate}
  \item check for signature
  \item exit if signature not found
  \item read start address
  \item create temporary copy of start address
  \item read amount of words to load
  \item loop: read all words and store them to RAM
  \item jump to start address
  \item execution loop
\end{enumerate}

\section{Computation theory}

\subsection{Mealy automaton}

current\_state = f(prev\_state) \\
output = out(input, current\_state)

\subsection{Moore automaton}

current\_state = f(prev\_state) \\
output = out(current\_state)

\section{Abstraction}

\begin{enumerate}
  \item User
  \item Application
  \item Operating system
  \item Architectural
  \item Register transfer
  \item Logical
  \item Electrical
  \item Physical
\end{enumerate}

\section{Pipelining}

\subsection{Dependencies}

\begin{description}
  \item[Data dependency] If the first instruction is storing a value to a
                         register which is used in the second instruction,
                         the result is not going to be ready on time.
  \item[Control dependency] If the first instruction includes a conditional
                         jump, the following instruction might not be the
                         one to be executed.
\end{description}

\subsection{Possible solutions to pipelining chaos}

\begin{itemize}
  \item Compiler adds ''No operation'' instructions
  \item Reorder machine instructions to avoid dependencies
\end{itemize}

\section{Caches}

\subsection{Replacement policies}

\begin{description}
  \item[LRU] Overwrite the least recently used element.
  \item[FIFO] Overwrite the oldest element.
  \item[LFU] Overwrite the least frequently used element.
\end{description}

\section{Questions}

Questions, you should be able to answer:

\begin{enumerate}
  \item Which registers does x86 define? How can you address the first
        8 bits of a register? What are EFLAGS?
  \item Which syntaxi are in use to write x86 assembler? What does the
        suffix ''l'' stand for? What is a word? What do the various modes
        look like syntactically?
  \item How is an array stored at hardware level? Explain the following
        assembler directives:

        \begin{itemize}
          \item .string
          \item .data
          \item .space
          \item .file
          \item .rept / .endr
          \item .long
          \item .globl
          \item .text
          \item .byte
          \item .equ
          \item \_start
        \end{itemize}

  \item Explain the ''-a'' and ''-gstabs'' options of GNU assembler
        (''as'' on CLI). What parts of the ASM source code will occur
        at the \emph{defined} and the \emph{undefined} symbols section.
  \item What's the 10-complement of 89, 9 and 2 (each with 2 digits)?
        What the 2-complement of 15, 7 and 5 (with 4 digits)?
  \item What's the difference between (5 bytes) and (1 long with a word)
        at hardware layer?
  \item Describe the procedure of a function call with a stack dump.
        How are function parameters and local variables assigned?
        What are the commands push, pop, call and ret for?
  \item Describe the following assembler operations:

        \begin{itemize}
          \item movl
          \item addl
          \item decl
          \item cmpl
          \item call
          \item ret
          \item jnz
          \item jz
          \item jge
          \item js
          \item jmp
        \end{itemize}

    \item What is each of the following programs doing in the compilation
          or software development process? readelf, objdump, ar, ranlib, ld,
          gdb, gcc.
    \item Name a command to link several object files to an executable.
    \item How can state be stored at hardware level? Explain it using a diagram
          of a latch and a flip flop.
    \item Which software and hardware layers do you know?
    \item Describe the fetch and execute algorithm.
    \item Describe the dependencies making work with pipelining difficult
          and discuss possible solutions.
    \item What are caches? Which cache size is recommended? Describe the
          principle of locality. Describe possible replacement policies.
          What is direct and associative mapping?
    \item Describe the idea of a DMA.
\end{enumerate}

\end{document}
