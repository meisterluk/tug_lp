\documentclass[a4paper]{report}
\usepackage[utf8]{inputenc}
\usepackage[pdfborder=0 0 0 0]{hyperref}
\usepackage{graphicx}
\usepackage{amsmath}

\title{
    Summary of course Software architecture \\
    \small{\url{http://kmi.tugraz.at/staff/denis/courses/sa/index.html}}
}
\author{Lukas Prokop}
\date{8th of Dec 2011}

\begin{document}
\maketitle
\tableofcontents

\chapter{Software architecture}

The following summary shall give you a general framework how to progress
from a software analysis process over a design phase to a final
implementation goal. However, in difference to courses like OAD, we will
focus on the various architectures which can be chosen for the
implementation.

\section{Requirements}

In the first phase we will receive customer wishes which are informal and
unprecise. It's our task to formalize them and define corresponding
requirements.

\subsection{Requirements}

There are three types of requirements:

\begin{itemize}
  \item non-functional (quality attributes, runtime-related or not)
  \item functional (structured language, use cases, description by
            formal methods like state diagrams)
  \item contextual (technology to be used, law / expertise / customer needs
            / customer experience)
\end{itemize}

Each of the non-functional requirements addresses one of the quality
attributes, which are described in two different lists: MeTRiCS and
PURS. Please remember that we are not talking about components, classes
and design decisions yet. Specifically quality attributes are not component
specific.

Using those requirements addressing quality attributes, we can identify
the key concepts of the application domain. Identifying it will help us
to keep attention to the customer's wishes and understanding his
requirements.

Those requirements can be defined in multiple iterations.

\subsection{PURS}

\begin{itemize}
  \item Performance
    \begin{itemize}
      \item Database design
      \item Choice of algorithms
      \item Communication
      \item Resource management
    \end{itemize}

  \item Usability
  \item Reliability
    \begin{itemize}
      \item MTTF: mean time to failure
      \item MTBF: mean time between failures
      \item MTTR: mean time to repair
      \item For web systems: $\text{MTTF} \to\infty$
      \item Acceptability of bugs depends
    \end{itemize}

  \item security
    \begin{itemize}
      \item Authentication and Authorization
      \item ACL (Access Control List)
      \item RBAC (Role-Based Access Control)
        \begin{itemize}
          \item resolves many-to-many relationship between user \& person
        \end{itemize}
    \end{itemize}
\end{itemize}

\subsection{MeTRiCS}

\begin{itemize}
  \item Maintainability
    \begin{itemize}
      \item Code comments
      \item OOP design principles
      \item consistency
      \item documentation
      \item Don't be afraid of refactoring, redesigning and rewriting!
      \item throw-away prototypes
      \item keep innovative
    \end{itemize}
  \item Evolvability
  \item Testability
  \item Reusability
  \item Integratability
  \item Configurability
  \item Scalability
\end{itemize}

\section{Analysis}

Requirements $\Rightarrow$ Key concepts $\Rightarrow$ application domain.

\section{Design}

Design consists of creating models. There 2 kinds of models:

\begin{itemize}
  \item Structural model \\
        is a static model of the system (ie. division of the system into
        components and connectors).
  \item Behavior model \\
        is a dynamic model of the system (ie. component interactions)
\end{itemize}

Functionality is what the system can do and behavior is the activity
sequence. Each component has a set of responsibilities.
Behavior is the way how these responsibilities are exercised to respond to
some event. An event may be an action of the user or an event from an
external system. A particular behavior is an event plus a response in the
form of a sequence of component responsibilities.

Design can be codified by box-and-lines diagrams and comparing the model to
OOP classes.

There are two major approaches how to identify components:

\begin{description}
  \item[noun method] underline keywords (ie. key concepts)
  \item[categorization] of possible concepts in
    \{
        Data\footnote{one component},
        Function\footnote{many components},
        Stakeholder\footnote{no components}, 
        system\footnote{external systems; are probably 1 component}, 
        hardware\footnote{one physical component}, 
        abstract concept\footnote{rarely components}
    \}
\end{description}

\section{Architectural views}

\begin{description}
  \item[Conceptual] Components are sets of responsibilities and connectors
                    are flows of informations.
  \item[Executional] Components are execution units (processes) and
                     connectors are messages between processes.
  \item[Implementation] Components are libraries, source code, \dots and
                        connectors are protocols \& APIs
\end{description}

\section{Methodology}

A methodology for software development shall be chosen after the definition
of requirements but before the software design process.

\begin{itemize}
  \item rational
  \item spiral
  \item agile
  \item evolutionary rapid
\end{itemize}




\chapter{Execution architecture}

Many quality attributes are explicitly addressed by the execution
architecture. Execution architecture focuses on the system structure
(eg. performance, security, reliability) and addresses concurrent
components, which are always simply summed up to components up to now.
In a single-computer, single-process and single-thread environment, the
execution architecture is very simple.

Executation architecture becomes very important whenever you design
network-based systems, multi-core or multiprocessing units, multi-threaded
systems or event-based GUIs. Components are considered to be either
hardware, concurrent subsystems, processes or threads. Connectors indicate
either synchronous, asynchronous calls or callback mechanisms.

Each model in Executation architecture is a model at a specific level of
granularity. Subsystems can be quite complex.

\section{Concurrent subsystems}

\begin{itemize}
  \item Always long-lived
  \item Creation means start of the service,
        Closing means shutdown of the service
  \item Operates through lifetime
\end{itemize}

\section{Process model}

\begin{itemize}
  \item Restricting a concurrency model to processes
  \item depicts the execution structure
  \item Do not go into details on external subsystems
\end{itemize}

\section{Execution stereotypes}

\begin{description}
  \item[user initiated]
    This stereotype describe systems where the events are controlled by the
    user. All UI components are always of this kind of stereotype. There
    might be a separate even thread listening to user input events.
  \item[active]
    components generate activity internally. This might be eg. a crawler of a
    search engine.
  \item[services]
    wait for requests of other components. They typically perform complex
    tasks and have a clear communication protocol. Example includes
    databases and the web.
\end{description}

\section{Comparison to CA}

There are \emph{no rules} how to decide where conceptual components reside
in the execution architecture (described by term ''Binding execution and
conceptual models'').

\begin{table}[h]
  \begin{center}
    \begin{tabular}{lll}
      Element & Conceptual architecture & Execution architecture \\
    \hline \hline
      Components & Domain-level responsibilities & Unit of concurrent
      activity \\

      Connectors & Information flow & Invocation \\
      Views & Single & Multiple \\
    \end{tabular}
  \end{center}
\end{table}





\chapter{Implementation architecture}

IA focuses on \emph{how} the system is built. It defines a set of
technological elements for the implementation. This includes software
packages, libraries, frameworks and classes. IA is a number of
implementation models. It addresses non-runtime quality attributes.

Components and connectors reflect software entities and relationships
(connectors describe a ''uses'' relationship). Each IA focuses on one of the
concurrent subsystems or processes from the execution view.

\section{Application components}

An application component is implementing domain-level responsibilities,
which can be found in the conceptual architecture. But also a number of
conceptual components can be mapped to a single application component.
Realization takes place by source code, files or binaries. UI components
typically map onto a single application component.

\section{Infrastructure components}

Infrastructure components are necessary for the running system, but do not
relate to the application functionality. Sometimes they act as an container
for several application components.

A container component provides an execution environment for the contained
components. So typical container components include frameworks, servers,
generic clients and off-the-shelf components.

''Impact Maps'' might help to illustrate dependencies between components.

\section{Connectors}

Connectors in an IA are API calls, callbacks, network protocols and signals.

\section{Infrastructure selection}

The selection of infrastructure is heavily influenced by the different
philosophies of the development team members.

\begin{itemize}
  \item Conceptual issues: MVC, component-based, service-based
  \item Executational issues: Spawning of external components
  \item Implementation issues: programming languages
  \item Contextual issues: Commercial products, OpenSource
  \item Organizational issues: Know-How, Team
\end{itemize}

\subsection{Algorithm (''Weighted Scoring method'')}

For each of those issues you have to identify and weigh criteria.
There are $n$ alternatives ($A_1\ldots A_n$) and $m$ different criteria
($C_1\ldots C_m$). Each alternative is for each criterion with score
$S_{ij}$. Each criterion has a weight relative to its importance
$W_1\ldots W_m$. The final score for $A_i$ is:

\[
    S(A_i) = \sum_{j=1}^m S_{ij} \cdot W_j
\]

\section{IA definition procedure}

\begin{enumerate}
  \item Find application components
  \item Find infrastructure components
  \item Design interfaces
  \item Define behavior design and verification
\end{enumerate}

We need a clear interface between all components. In best case, those
interface are already standardized (eg. HTTP). In behavior design we have
to go into detail for IA and use case maps not precise enough for it;
sequence diagrams are recommended.

Prototypes will help us to verify our final IA.


\begin{table}[h]
  \begin{center}
    \begin{tabular}{llll}
      Element & CA & EA & IA \\
    \hline \hline
      Components & Domain-level resp. & Unit of concurrent activity & Impl. module \\

      Connectors & Information flow & Invocation & ''Uses'' relationship \\
      Views & Single & Multiple & Split \\
    \end{tabular}
  \end{center}
\end{table}




\chapter{OO Design principles}

Design is the process of modeling new software entities to satisfy
requirements. Software entities might be different things depending on the
level of granularity and programming paradigm. Design is not an exact
entity and therefore not measureable.

\begin{table}[h]
  \begin{center}
    \begin{tabular}{llll}
    \hline
      Procedural    & OOP \\
    \hline \hline
      producedures  & classes \\
      variables     & objects \\
                    & relationships \\
    \end{tabular}
  \end{center}
\end{table}

We have discuss 4 OO design principles:

\begin{enumerate}
  \item Open-Closed
  \item ''Design by contract'' or ''Liskov substitution principle''
  \item Single responsibility principle
  \item Law of Demeter
\end{enumerate}

\section{Open-Closed}

''Software should be open for extension but closed for modification''

\begin{itemize}
  \item You should be able to extend a class without modifying it
  \item Reason for this principle: abstraction (fixed design but limitless
        behaviors)
  \item Early design decision: Fixed vs expandable system parts
        (clear interfaces)
  \item We can test components finally and extend them without modifications
  \item Example violation of this rule: usage of \texttt{instanceof} operator
        (destroys polymorphism)
\end{itemize}

\section{Design by Contract}

''Define a contract between a class and its users''

\begin{itemize}
  \item Preconditions and postconditions for methods (you might want to
        insert tests for those conditions using \texttt{assert}) \\
        Invariants for the class (technical view of pre- and postconditions)
  \item Liskow Substitution Principle (related to ''Design by Contract'')
        \begin{itemize}
          \item ''If for each object o1 of type S there is an object o2 of
                  type T such that for all programs P defined in terms of T,
                  the behavior of P is unchanged when o1 is substituted for
                  o2 then S is a subtype of T.''
          \item Polymorphism does it anyway, but we are talking about
                implemented methods in the sub- and superclass
          \item Example violation of this rule:
                return value is null object and no real object.
          \item Allowed within this rule:
                Weaken preconditions, strengthen postconditions
          \item Cannot satisfy this rule? Refactor class hierarchy!
        \end{itemize}
\end{itemize}

\section{Single responsibility principle}

''One entity (class, object), one responsibility'' (separation of concerns)

\begin{itemize}
  \item A responsibility is in fact a reason to change
  \item Example violation of this rule: Student is reading from DB
        \emph{and} printing HTML
\end{itemize}

\section{Law of Demeter}

''A method of an object should only invoke methods of itself, its
parameters, objects it creates and its members''

\begin{itemize}
  \item Chaining of method calls across various objects is evil.
\end{itemize}



\chapter{Architectural styles}

An architectural style is a particular pattern that focuses on the
large-scale structure of a system typically described by the common
vocabulary.

Variants of architectural style:

\begin{itemize}
  \item Data-centered architecture
  \item Data-flow architecture
  \item Abstraction-layer architecture
  \item N-tier architecture
  \item Notification architecture
  \item Remote invocation and service architecture
  \item Heterogeneous architecture
\end{itemize}

\section{Data-centered architecture (DCA)}

Systems in which the access and update of a widely accessed data store is the
primary goal. Basically it is just a centralized data storage with a number
of clients. This architecture requires three protocols: communication, data
definition and data manipulation. This architecture has two subtypes:

\begin{description}
  \item[Repository] client sends a request, system executes
  \item[Blackboard] system spread notifications / data to subscribers
\end{description}

\begin{itemize}
  \item ensures data integrity
  \item reliable, secure
  \item testability guaranteed
  \item Clients are independent $\rightarrow$ performance and usability
        at clientside good
  \item problems with scalability
    \begin{itemize}
      \item shared repositories
      \item replication; but increases complexity
    \end{itemize}
\end{itemize}

Example implementations:

\begin{enumerate}
  \item RDBMS (repository with database schemas and SQL for communication)
  \item Web architecture (hypermedia data model, pages with links, HTTP
        for communication, integrity not guaranteed [404 error], extremely
        scalable)
\end{enumerate}

See also: resource oriented architectures

\section{Data-flow architectures (DFA)}

An architectural style to improve the quality of reuse and modifiability.
We are viewing the system as a sequence of transformations on successive
pieces of input data. Finally this data is assigned to its destination.
There are two variations:

\begin{description}
  \item[structural variation]
    more complex topologies like loops, branches, more than 1 input
  \item[communicational variation]
    synchronous behavior
\end{description}

\begin{itemize}
  \item Maintainability
  \item Modularity
\end{itemize}

Example implementations:

\begin{enumerate}
  \item UNIX pipes (pipes for communication, probably concurrent processing of incremental data)
  \item any kind of filter
\end{enumerate}

\section{Abstract layer architectures}

Abstract layer architectures introduce a layering system where each layer is
on top of another one and there are well-defined interfaces between them.

\begin{itemize}
  \item reduces complexity
  \item improves modularity
  \item reusability
  \item maintainability
\end{itemize}

Example implementations:

\begin{enumerate}
  \item Operating systems
  \item Virtual machine (interface between compiler \& real machine,
          improves portability)
  \item Network protocol stack
\end{enumerate}

\section{N-tier architectures}

N-tier architectures evolved from business applications and can nowadays be
widely seen in client-server models for the web. They try to separate
presentation, application and data storage components.

\begin{description}
  \item[2-tier]
    Rich clients run application and server stores data
  \item[3-tier]
    Rich clients, application server and data server
  \item[Rich clients]
    Clients which have full knowledge of application.
    Either they implement a standard application and/or protocol
    or a custom one.
  \item[Thin clients]
    Small clients like in X-server architecture of UNIX-based operating
    systems
\end{description}

\section{Notifications architecture}

Information and activity gets propagated by a notification mechanism
(listener and callbacks). It's basically a more abstract view of the
blackboard data-centric architectures. Clients can ''register'' to receive
notifications and receive data either when data is relevant for them
(''interested user'') or always (''broadcast'').

Example implementations:

\begin{enumerate}
  \item GUI event handling
\end{enumerate}

\section{Remote invocation and service architectures}

This architecture involves distributed processing components and a client
invokes a function on a remote component.

\begin{itemize}
  \item Increased performance through distributed computation
  \item Tightly coupled components
\end{itemize}

\emph{Service architectures} store where services are registered.

\section{Heterogeneous architectures}

No real system follows strictly only a single style. Therefore architectures
might be structurally heterogeneous:

\begin{itemize}
  \item Overall architecture follows one style
  \item Single components follows other style
\end{itemize}

Example implementations:

\begin{itemize}
  \item Web has 2-tier architecture (browser and server) and browser has
        notification architecture (user events)
  \item Web-based search engine \\
        Conceptually: data-centric, layered, 3-tier \\
        Structurally: layered (network), 3-tier, notification \\
        Execution: distributed, service-oriented, \dots
\end{itemize}

\chapter{Web architecture}

\emph{Two views:} Is the web an application platform or a huge distributed
database?

The Web started as a static information system. \emph{Hypermedia} are
documents linked into a web. The user is able to retrieve documents by
following uni-directional links. It's major advantage is that it only
requires simple components and everything is based on HTTP, HTML and
URL standards (cross-platform, global scope).

\begin{quote}
    Web's major goal was to be a shared information space
    through which people and machines could communicate \\
    ---Tim Berners Lee
\end{quote}

\begin{itemize}
  \item Usability: easy to create, structure and reference information.
        voluntary participation, very error forgiving
  \item Simple: easy implementation, text-based components, only a
        ''simple'' HTML parser required
  \item Extensibility: requirements changed, forms were introduced
  \item Scalability: anarchic, dezentralized
\end{itemize}

Constraints were introduced on the web architecture to obtain an optimal
solution to the requirements and quality attributes.

\begin{itemize}
  \item Client-server model (separation of concerns)
  \item stateless architecture (client-side sessions)
    \begin{itemize}
      \item Improves visibility, reliability, scalability
    \end{itemize}
  \item information can be labeled as cacheable
    \begin{itemize}
      \item Improves efficiency, scalability, user-perceived performance
    \end{itemize}
  \item Uniform interface
    \begin{itemize}
      \item Identification of resources (URL)
      \item manipulation of resources through representations (HTML/XML)
      \item self-descriptive message (GET, POST, PUT, DELETE)
      \item layered system (scalability by proxies, shared-caches and
            gateways; also load-balancing)
      \item Code on demand (Java applets, flash, Javascript)
    \end{itemize}
\end{itemize}

Web's specifics:

\begin{enumerate}
  \item User requirements
  \item User interface and usability
  \item Application state and hypertext
  \item Addressability
  \item Architecture
\end{enumerate}

\section{Application state on the Web: Hypertext}

\begin{itemize}
  \item HTTP is stateless (connection-less)
  \item Application logic manages sessions
  \item AJAX problems: recovery of states in new session impossible, browser
            back button problem
  \item server state vs client state = scalability problems vs recovery
            problems
  \item Eg. Google Maps: permalinks for states (second addressbar)
\end{itemize}

\section{Addressability}

\begin{itemize}
  \item URLs: good usability and meaninful links
  \item bookmarks
  \item linking of services is possible
\end{itemize}

\section{User-oriented DB applications}

\begin{itemize}
  \item traditionally built with N-tier architecture (started with $N=2$)
  \item DBMS layer (Relational Database) and app/presentation (scripts) layer \\
        Major issue: Changes in application logic leads to changes in
        presentation functionality
  \item Solution is a 3-layer system:
        Separation between Application and Presentation \\
        introduction of AJAX $\rightarrow$ application logic either on
        client or server
  \item MVC models since GUIs got introduced \\
        (especially popular in web applications)
\end{itemize}

\section{Data management}

\begin{itemize}
  \item Data Access Object
  \item ORM (Object Relational Mapper)
  \item Information retrieval by fulltext search and/or link analysis
  \item Metadata (taxonomy/folksonomy)
  \item Data in Web is organized in simple node-link model
  \item Each node is a data item with a unique address and representation
  \item Programmers combine a number of services to achieve a desired
        functionality and create a distributed system (eg. mashups)
\end{itemize}

\subsection{User-centric view}

What is Google? A \emph{service} to query a massive database (Web search
index). \\
What is web application? A \emph{service} offering a specific functionality
(remotely) \\
What is a web site? A \emph{service} offering specific human consumable
information.

\subsection{The Programmable Web}

\begin{quote}
    ''Programmers follow the architectural style to integrate and combine
    services to achieve a desired functionality''
\end{quote}

HTTP is used for data transportation, XML for data representation, but
services also offer HTML, JSON, plain text and binary data.

\subsubsection{Classification}

How to transmit method information (what to do):

\begin{itemize}
  \item HTTP methods (standardized, not extensible)
  \item URL/GET parameter (not standardized, extensible, eg. Flickr)
  \item SOAP and WSDL
\end{itemize}

How to transmit scoping information (where to do):

\begin{itemize}
  \item URL/GET parameter
  \item SOAP
\end{itemize}

\subsubsection{Competing architectures in practice}

\begin{itemize}
  \item RESTful, Resource-Oriented architectures
  \item RPC-style architectures
  \item REST-RPC hybrid architectures
\end{itemize}

\section{Remote Procedure Call (RPC)}

Server receives a request envelope with all necessary data to perform a
certain task. Server sends back a similar envelope.

Protocols: SOAP, XML-RPC

Problems:

\begin{itemize}
  \item RPC implies an API
  \item APIs tend to enforce a tight coupling of modules and systems
  \item Where is all the nice Web architecture gone?
\end{itemize}

\section{Resource Oriented Architectures}

\begin{itemize}
  \item method information in HTTP method
  \item scoping information in URL
\end{itemize}

4 features of ROA (resource oriented architectures):

\begin{enumerate}
  \item addressability
  \item statelessness
  \item uniform interface
  \item connectedness
\end{enumerate}

\emph{Idempotence:} same operation has same effect however often it is applied.

\section{Designing RESTful services}

REST originated from a PhD thesis by Roy Fielding.

\begin{enumerate}
  \item Figure out data set
  \item Split data set into resources
  \item For each resource: define URL
  \item Expose a subset of the uniform interface
  \item Design representations accepted from the client
  \item Design representations served to the client
  \item Integrate this resource into other resources using links
  \item Consider possible application states
  \item Consider possible error states
\end{enumerate}

Software packages: Java Restlet, Plugin for ruby on rails, Django in python

\end{document}
