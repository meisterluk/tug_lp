\documentclass[11pt]{article}
\usepackage{euler}
\usepackage{xltxtra}
\usepackage{amsmath}
%\usepackage{amssymb}
\usepackage[pdfborder={0 0 0}]{hyperref}
\usepackage{listings}
%\setmainfont[Mapping=tex-text]{Ubuntu}
%{FreeSerif}
%{DejaVu Serif}

\date{26th of June 2012}
\author{Lukas Prokop}
\title{Software paradigms exam 27.6.2011}

% Definitions by Daniel Gruß
% https://github.com/lava2/swp/blob/master/commands.tex
% various operators
\DeclareMathOperator {\id}{id}
\DeclareMathOperator*{\ggt}{ggT}
\DeclareMathOperator*{\kgv}{kgV}
\DeclareMathOperator {\ord}{ord}
\DeclareMathOperator {\fmod}{mod}
\DeclareMathOperator*{\trace}{tr}
\DeclareMathOperator {\nil}{nil}
\DeclareMathOperator*{\supp}{Supp}
\DeclareMathOperator {\lm}{lm}
\DeclareMathOperator {\lt}{lt}
\DeclareMathOperator {\lc}{lc}
\DeclareMathOperator*{\rest}{rest}
\DeclareMathOperator*{\tmult}{\:{mult}\:}
\DeclareMathOperator*{\tplus}{\:{plus}\:}
\DeclareMathOperator*{\tminus}{\:{minus}\:}
\DeclareMathOperator*{\tnull}{\:{null}\:}
\DeclareMathOperator*{\teins}{\:{eins}\:}
\DeclareMathOperator*{\tif}{\:{if}\:}
\DeclareMathOperator*{\telse}{\:{else}\:}
\DeclareMathOperator*{\tthen}{\:{then}\:}
\DeclareMathOperator*{\tlt}{\:{lt?}\:}
\DeclareMathOperator*{\teq}{\:{eq?}\:}
\DeclareMathOperator*{\tumult}{\:\u{mult}\:}
\DeclareMathOperator*{\tuplus}{\:\u{plus}\:}
\DeclareMathOperator*{\tuminus}{\:\u{minus}\:}
\DeclareMathOperator*{\tunull}{\:\u{null}\:}
\DeclareMathOperator*{\tueins}{\:\u{eins}\:}
\DeclareMathOperator*{\tuif}{\:\u{if}\:}
\DeclareMathOperator*{\tuelse}{\:\u{else}\:}
\DeclareMathOperator*{\tuthen}{\:\u{then}\:}
\DeclareMathOperator*{\tult}{\:\u{lt?}\:}
\DeclareMathOperator*{\tueq}{\:\u{eq?}\:}
\DeclareMathOperator {\End}{end}
\DeclareMathOperator*{\rang}{rang}
\DeclareMathOperator*{\dist}{d}
\DeclareMathOperator*{\real}{Re}
\DeclareMathOperator*{\imag}{Im}
\DeclareMathOperator*{\sign}{sign}
\DeclareMathOperator*{\weight}{weight}

% graphics and sectioning
\newcommand{\skizze}[1]{$ $ \begin{center}\includegraphics[scale = 0.33]{images/#1.png}\end{center}}
\newcommand{\mychapter}[1]{\chapter{#1}}
\newcommand{\mysection}[1]{\section{#1}}
\newcommand{\mysubsection}[1]{\subsection{#1}}
\newcommand{\separate}{\begin{center}*\:*\:*\end{center}}

% definition and proof tools
\newcommand{\proofdone}{\hfill \ensuremath{\Box}}
\newcommand{\mathdone}{\tag*{\ensuremath{\Box}}}

\newcommand{\defas}{\mathrel{\mathop:}=}
\newcommand{\asdef}{=\mathrel{\mathop:}}

\newcommand{\there}{``$\Rightarrow$'' \;}
\newcommand{\back}{``$\Leftarrow$'' \;}

\newcommand{\then}{\: \Rightarrow \:}
\newcommand{\ifonlyif}{\: \Leftrightarrow \:}
\renewcommand{\u}[1]{\underline{#1}}
\newcommand{\tu}[1]{\text{\underline{#1}}}

\newcommand{\tobe}{\overset{!}{=}}
\newcommand{\isit}{\overset{?}{=}}

% referencing
\newcommand{\tagstar}{\tag*{$\star$}}
\newcommand{\refstar}{\ensuremath{(\star)}}
\newcommand{\alrefstar}{\tag*{(\star)}}
\newcommand{\tagstarn}[1]{\tag*{$\star_{#1}$}}
\newcommand{\refstarn}[1]{\ensuremath{(\star_{#1})}}
\newcommand{\alrefstarn}[1]{\tag*{(\star_{#1})}}

% mathtext
\newcommand{\sonst}{\text{sonst}}
\newcommand{\const}{\mathrm{const}}

% braces
\newcommand{\abs}[1]{\left\lvert{#1}\right\rvert}
\newcommand{\norm}[1]{\left\lVert{#1}\right\rVert}
\newcommand{\angled}[1]{\left\langle{#1}\right\rangle}
\newcommand{\tangled}[1]{\langle{#1}\rangle}
\newcommand{\braced}[1]{\left( {#1} \right)}
\newcommand{\sbraced}[1]{\left\{#1\right\}}
\newcommand{\bbraced}[1]{\left[#1\right]}

\newcommand{\bbinom}[2]{\genfrac[]{0pt}{0}{#1}{#2}}
\newcommand{\cbinom}[2]{\genfrac{\{}{\}}{0pt}{0}{#1}{#2}}
\newcommand{\ebinom}[2]{\genfrac{}{}{0pt}{0}{#1}{#2}}

\newcommand{\gausslo}[1]{\left\lfloor {#1} \right\rfloor}
\newcommand{\gausshi}[1]{\left\lceil {#1} \right\rceil}
\newcommand{\gaussround}[1]{\left\lfloor {#1} \right\rceil}

% under/overlining
\newcommand{\ov}[1]{\overline{#1}}
\newcommand{\un}[1]{\underline{#1}}
\newcommand{\vek}[1]{\vec{#1}}

% relations
\newcommand{\sm}{\setminus}
\newcommand{\se}{\subseteq}
\newcommand{\nse}{\nsubseteq}
\newcommand{\sne}{\nsubseteq}
\newcommand{\ise}{\supseteq}
\newcommand{\nise}{\nsupseteq}
\newcommand{\sneq}{\subsetneqq}
\newcommand{\isneq}{\supsetneqq}

\newcommand{\subideal}{\trianglelefteq}
\newcommand{\isubideal}{\trianglerighteq}

\newcommand{\divs}{\mathrel\vert}
\newcommand{\ndivs}{\nmid}
\newcommand{\iso}{\simeq}

\newcommand{\nequiv}{\not\equiv}

% sets
\newcommand{\N}{\mathbb{N}}
\newcommand{\Z}{\mathbb{Z}}
\newcommand{\Q}{\mathbb{Q}}
\newcommand{\R}{\mathbb{R}}
\newcommand{\C}{\mathbb{C}}
\newcommand{\F}{\mathbb{F}}
\newcommand{\K}{\mathbb{K}}
\newcommand{\primes}{\mathbb{P}}
\newcommand{\potenz}{\mathcal{P}}

\newcommand{\dotcup}{\ensuremath{\mathaccent\cdot\cup}}

\newcommand{\prob}{\mathbb{P}}
\newcommand{\erw}{\mathbb{E}}
\newcommand{\variance}{\mathbb{V}}

% greek symbols
\newcommand{\fee}{\varphi}
\newcommand{\eps}{\varepsilon}

% various tools
\newcommand{\where}{\mathrel\big\vert}
\newcommand{\limited}{\big\vert}

% complexity theory
\newcommand{\ordn}{\mathcal{O}}
\newcommand{\oh}{\mathcal{O}}

% analysis
\newcommand{\de}{\, \dd}
\newcommand{\dd}{\mathrm{d}}

\renewcommand{\thesubsection}{\arabic{subsection}}
\newcommand{\rbr}[1]{\left(#1\right)}
\newcommand{\ebr}[1]{\left[#1\right]}
\newcommand{\sbr}[1]{\left<#1\right>}
\newcommand{\gbr}[1]{\left\{#1\right\}}
\renewcommand{\det}[1]{\text{det}\left(#1\right)}
\renewcommand{\u}[1]{\underline{#1}}
\renewcommand{\o}[1]{\overline{#1}}
\renewcommand{\b}[1]{\textbf{#1}}

\newcommand{\E}{\mathbb{E}}
\renewcommand{\P}{\mathbb{P}}
%\newcommand{\I}{\mathbb{I}}
\newcommand{\V}{\mathbb{V}}
\newcommand{\dx}{\text{d}x}
\newcommand{\D}{\mathbb{D}}
\newcommand{\kA}{\mathcal{A}}
\newcommand{\kC}{\mathcal{C}}
\newcommand{\kP}{\mathcal{P}}

\newcommand{\Lra}{\:\Leftrightarrow\:}
\newcommand{\lra}{\:\leftrightarrow\:}
\newcommand{\Ra}{\:\Rightarrow\:}
\newcommand{\ra}{\:\rightarrow\:}
\newcommand{\La}{\:\Leftarrow\:}
\newcommand{\la}{\:\leftarrow\:}

\newcommand{\limi}[1]{\lim_{#1 \rightarrow \infty}}
\newcommand{\mceil}[1]{\left\lceil #1 \right\rceil}
\newcommand{\mfloor}[1]{\left\lfloor #1 \right\rfloor}

\DeclareMathOperator{\ggT}{ggT}
\DeclareMathOperator{\kgV}{kgV}

\DeclareMathOperator{\RR}{R}
\DeclareMathOperator{\notdivides}{\nmid}

\newcommand{\lng}[1]{\mathcal{#1}}
\newcommand{\kwu}[1]{\text{\:\u{#1}\:}}
\newcommand{\kw}[1]{\text{\:{#1}\:}}
\newcommand{\I}[3]{I_{\lng{#1}}\left(#2,\tu{#3}\right)}
\newcommand{\Iu}[3]{I_{\lng{#1}}\left(#2,#3\right)}

\newcommand{\twedge}{\text{$\wedge$}}
\newcommand{\tforall}{\text{$\forall$}}
\newcommand{\texists}{\text{$\exists$}}
\newcommand{\tvee}{\text{$\vee$}}
\newcommand{\tto}{\text{$\to$}}
\newcommand{\tneg}{\text{$\neg$}}
% End Definitions

\DeclareMathOperator*{\s}{s}
\DeclareMathOperator*{\binStack}{binStack}
\DeclareMathOperator*{\add}{add}
\DeclareMathOperator*{\len}{len}
\newcommand{\Null}{\text{null}}
\newcommand{\set}[1]{\left\{#1\right\}}

\lstdefinelanguage{EXP}%
  {keywords={if,then,else},%
%   emph={plus,minus,mult,add0,add1,sub,build,rest,first,second,reverse,depth,max,nth,lt?,eq?,is0?,is1?,isEmpty?,atom?},
   moreemph={null,eins,0,1,[]},% epsilon is missing
   emphstyle=\emph,
   mathescape={true}%
  }%
\newcommand{\NE}{\text{NE:}\hspace{5pt}}
\newcommand{\NR}{\text{NR:}\hspace{5pt}}

\begin{document}
\maketitle
\tableofcontents
\newpage
%
\section{Exercise 1}
%
%\begin{table}
  \begin{tabular}{lcccc}
           &    S     &    A    &   B   &   C \\ \hline
    \textsc{FIRST}  & $\set{\u{a}}$ & $\set{\u{a}}$ & $\set{\u{b}, \u{c}, \u{a}}$ & $\set{\u{c}, \varepsilon}$ \\
    \textsc{FOLLOW} & $\set{\$}$ & $\set{\$, \u{b}, \u{c}, \u{a}}$ & $\set{\$, \u{b}, \u{c}, \u{a}}$ & $\set{\u{a}}$ \\
  \end{tabular}
%\end{table}
%
\section{Exercise 2a}
%
%\begin{table}
  \begin{tabular}{lcccc}
      &    $\u{a}$     &    $\u{b}$    &   $\u{c}$   &   \$ \\
  \hline
    S & $S \to A$      & $S \to A$     & $S \to A$   & \\
    A & $A \to AB, A \to \u{a}$ &      &             & \\
    B & $B \to CA$     & $B \to \u{b}$ & $B \to CA$  & \\
    C & $C \to \varepsilon$ &          & $C \to \u{c}D, C \to \u{c}$ & \\
    D &                & $D \to \u{b}$ & $D \to \u{c}$ & \\
  \end{tabular}
%\end{table}

$(A, \u{a})$ contains a left recursion.
$(C, \u{c})$ contains an ambiguity.
%
\section{Exercise 2b}
%
\begin{align*}
  S &\to IR \\
  R &\to CIR \\
  R &\to \varepsilon \\[7pt]
  I &\to \u{a}J \\
  J &\to \u{b}J \\
  J &\to \varepsilon \\[7pt]
  C &\to \u{c}D \\
  C &\to \varepsilon \\
  D &\to \u{c} \\
  D &\to \u{b} \\
  D &\to \varepsilon
\end{align*}
%
%\begin{table}
 \begin{center}
  \begin{tabular}{lcccc}
      & $\u{a}$ & $\u{b}$ & $\u{c}$ & \$ \\
  \hline
    S & $S \to IR$ &      &         &  \\
    R & $R \to CIR$ &     & $R \to CIR$ & $R \to \epsilon$ \\
    I & $I \to \u{a}J$ &  &         &  \\
    J & $J \to \varepsilon$ & $J \to \u{b}J$ & $J \to \varepsilon$ & $J \to \varepsilon$ \\
    C & $C \to \varepsilon$ & & $C \to \u{c}D$ & \\
    D & $D \to \varepsilon$ & $D \to \u{b}$ & $D \to \u{c}$ & \\
  \end{tabular}
 \end{center}
%\end{table}
%
\section{Exercise 2c}
%
\begin{center}
 \begin{tabular}{lll}
  \textsc{Input}           & \textsc{Stack}        & \textsc{Comment} \\
  $\u{\$abcbba}$           & $S$                   & $S \to IR$ \\
  $\u{\$abcbba}$           & $RI$                  & $I \to \u{a}J$ \\
  $\u{\$abcbba}$           & $RJ\u{a}$             & $J \to \u{b}J$ \\
  $\u{\$abcbb}$            & $RJ\u{b}$             & $J \to \u{b}J$ \\
  $\u{\$abcb}$             & $RJ\u{b}$             & $J \to \varepsilon$ \\
  $\u{\$abc}$              & $R$                   & $R \to CIR$ \\
  $\u{\$abc}$              & $RIC$                 & $C \to \u{c}D$ \\
  $\u{\$abc}$              & $RID\u{c}$            & $D \to \u{b}$ \\
  $\u{\$ab}$               & $RI\u{b}$             & $I \to \u{a}J$ \\
  $\u{\$a}$                & $RJ\u{a}$             & $J \to \varepsilon$ \\
  $\u{\$}$                 & $R$                   & $R \to \varepsilon$ \\
  $\u{\$}$                 &                       & accepted \\
 \end{tabular}
\end{center}
%
\section{Exercise 3}
%
\lstset{language=EXP}
\begin{lstlisting}
factorial(x) =
  if eq?(x, 0) then eins
  else if eq?(x, eins) then eins
  else mult(factorial(minus(x, eins)), x)
\end{lstlisting}

Proof by complete induction.

\subsection{Induction hypothesis}
\[
  \forall \omega, \omega(\u{x}) \leq n, \omega(\u{x}) \in ℕ:
    I(\delta, \omega, \u{factorial(x)}) = \omega(\u{x})!
\]

\subsection{Induction base 0}
\[
  \omega(\u{x}) = 0
\] \[
  I(\delta, \omega, \u{\text{if eq?(x, 0) then eins else if eq?}}\ldots)
\] \[
  \NR I(\delta, \omega, \u{\text{eq?(x, 0)}})
    = \operatorname{eq?}(I(\delta, \omega, \u{x}), I(\delta, \omega, \u{0}))
    = \operatorname{eq?}(\omega(\u{x}), 0)
    = T
\] \[
  I(\delta, \omega, \u{\text{eins}}) = 1 = 0!
\]

\subsection{Induction base 1}
%
\[
  \omega(\u{x}) = 1
\] \[
  I(\delta, \omega, \u{\text{if eq?(x, 0) then eins else if eq?}}\ldots)
\] \[
  \NR I(\delta, \omega, \u{\text{eq?(x, 0)}})
    = \operatorname{eq?}(I(\delta, \omega, \u{x}), I(\delta, \omega, \u{0}))
    = \operatorname{eq?}(\omega(\u{x}), 0)
    = F
\] \[
  I(\delta, \omega, \u{\text{if eq?(x, 1) then eins else mult}}\ldots)
\] \[
  \NR I(\delta, \omega, \u{\text{eq?(x, eins)}})
    = \operatorname{eq?}(I(\delta, \omega, \u{x}), I(\delta, \omega, \u{\text{eins}}))
    = \operatorname{eq?}(\omega(\u{x}), 1)
    = T
\] \[
  I(\delta, \omega, \u{\text{eins}}\ldots) = 1 = 1!
\]

\subsection{Induction step}
%
\[
  \omega(\u{x}) = n + 1 \qquad n \geq 2
\] \[
  I(\delta, \omega, \u{\text{if eq?(x, 0) then eins else if eq?}}\ldots)
\] \[
  \NR I(\delta, \omega, \u{\text{eq?(x, 0)}})
    = \operatorname{eq?}(I(\delta, \omega, \u{x}), I(\delta, \omega, \u{0}))
    = \operatorname{eq?}(\omega(\u{x}), 0)
    = F
\] \[
  I(\delta, \omega, \u{\text{if eq?(x, 1) then eins else mult}}\ldots)
\] \[
  \NR I(\delta, \omega, \u{\text{eq?(x, eins)}})
    = \operatorname{eq?}(I(\delta, \omega, \u{x}), I(\delta, \omega, \u{\text{eins}}))
    = \operatorname{eq?}(\omega(\u{x}), 1)
    = F
\] \[
  I(\delta, \omega, \u{\text{mult(factorial(minus(x, eins)), x)}})
    = \operatorname{mult}(I(\delta, \omega, \u{\text{factorial(minus(x, eins))}}), I(\delta, \omega, \u{x}))
\] \[
  \NE \omega'(\u{x})
    = I(\delta, \omega, \u{\text{minus(x, eins)}})
    = \operatorname{minus}(I(\delta, \omega, \u{x}), I(\delta, \omega, \u{\text{eins}}))
\] \[
    = \operatorname{minus}(\omega(\u{x}), 1)
    = \operatorname{minus}(n + 1, 1) = n
\] \[
  \operatorname{mult}(n!, n+1) = (n+1)! \qquad \text{ corresponds to hypothesis}
\]

%
\section{Exercise 4a}
%
\lstset{language=prolog}
\begin{lstlisting}
odd(binStack(V, R)) :- odd(R).
odd(binStack(s(0), null)).
\end{lstlisting}
%
\section{Exercise 4b}
%
Tables~\ref{tab:4b-def} and~\ref{tab:4b-deriv}.
\begin{table}[h]
  \begin{tabular}{cl}
    Q   & $\neg \len(\binStack(0, \binStack(0, \binStack(0, \Null))), \s(\s(\s(0))))$ \\
    C1  & $\add(X, 0, X)$ \\
    C2  & $\add(X, \s(Y), \s(Z)) \lor \neg \add(X, Y, Z)$ \\
    C3  & $\len(\Null, 0)$ \\
    C4  & $\len(\binStack(V, R), E) \lor \neg \len(R, S) \lor \neg \add(\s(0), S, E)$
  \end{tabular}
  \caption{Rules of exercise 4b in logical notation}
  \label{tab:4b-def}
\end{table}
%
\begin{table}[h]
  \begin{tabular}{cl}
    Q   & $\neg \len(\binStack(0, \binStack(0, \binStack(0, \Null))), \s(\s(\s(0))))$ \\
    C4  & $\len(\binStack(V, R), E) \lor \neg \len(R, S) \lor \neg \add(\s(0), S, E)$ \\
    $\Theta$ & $\{ V = 0, R = \binStack(0, \binStack(0, null)), E = \s(\s(\s(0))) \}$ \\
    R   & $\neg \len(\binStack(0, \binStack(0, \Null)), S_1) \lor \neg \add(\s(0), S_1, \s(\s(\s(0))))$ \\[10pt]%

    C4  & $\len(\binStack(V, R), E) \lor \neg \len(R, S) \lor \neg \add(\s(0), S, E)$ \\
    $\Theta$ & $\{ V_2 = 0, R_2 = \binStack(0, \Null), E_2 = S_1 \}$ \\
    R   & $\neg \len(\binStack(0, \Null), S_2) \lor \neg \add(\s(0), S_2, S_1)
           \lor \neg \add(\s(0), S_1, \s(\s(\s(0))))$ \\[10pt]%

    C3  & $\len(\Null, 0)$ \\
    $\Theta$ & $\{ S_3 = 0 \}$ \\
    R   & $\neg \add(\s(0), 0, S_2) \lor \neg \add(\s(0), S_2, S_1) \lor \neg \add(\s(0), S_1, \s(\s(\s(0))))$ \\[10pt]%

    C1  & $\add(X, 0, X)$ \\
    $\Theta$ & $\{ S_2 = \s(0), X = S_2 \}$ \\
    R   & $\neg \add(\s(0), \s(0), S_1) \lor \neg \add(\s(0), S_1, \s(\s(\s(0))))$ \\[10pt]%

    C2  & $\add(X, \s(Y), \s(Z)) \lor \neg \add(X, Y, Z)$ \\
    $\Theta$ & $\{ X_2 = \s(0), Y_2 = 0, Z_2 = A, S_1 = \s(A) \}$ \\
    R   & $\neg \add(\s(0), 0, A) \lor \neg \add(\s(0), \s(A), \s(\s(\s(0))))$ \\[10pt]%

    C1  & $\add(X, 0, X)$ \\
    $\Theta$ & $\{ X_3 = \s(0), A = X_3 \}$ \\
    R   & $\neg \add(\s(0), \s(\s(0)), \s(\s(\s(0)))$ \\[10pt]%

    C2  & $\add(X, \s(Y), \s(Z)) \lor \neg \add(X, Y, Z)$ \\
    $\Theta$ & $\{ X_4 = \s(0), Y_3 = \s(0), Z_3 = \s(\s(0)) \}$ \\
    R   & $\neg \add(\s(0), \s(0), \s(\s(0)))$ \\[10pt]%

    C2  & $\add(X, \s(Y), \s(Z)) \lor \neg \add(X, Y, Z)$ \\
    $\Theta$ & $\{ X_5 = \s(0), Y_4 = 0, Z_4 = \s(0) \}$ \\
    R   & $\neg \add(\s(0), 0, \s(0))$ \\[10pt]%

    C1  & $\add(X, 0, X)$ \\
    $\Theta$ & $\{ X_6 = \s(0) \}$ \\
    R   & empty, query got successfully derivated. \\
  \end{tabular}
  \caption{Derivation of the query in exercise 4b}
  \label{tab:4b-deriv}
\end{table}
%
\section{Exercise 5}
%
\[
  \pi[\text{add}](\pi(a), \pi(b)) = \operatorname{add}(a, b)
\]
%
\lstset{language=EXP}
\begin{lstlisting}
add(a, b) =
  if eq?(mod(a, 2), 1) then
    if eq?(mod(b, 2), 1) then
      sub(add(a, b), 1)
    else
      if gt?(a, b) then
        sub(a, b)
      else
        add(sub(b, a), 1)
  else
    if eq?(mod(b, 2), 1) then
      if gt?(a, b) then
        add(sub(a, b), 1)
      else
        sub(b, a)
    else
      add(a, b)
\end{lstlisting}

\begin{enumerate}
  \item Sign of first parameter is positive, second parameter the same
  \item Sign of first parameter is positive, second parameter is negative
  \item Sign of first parameter is negative, second parameter is positive
  \item Sign of first parameter is negative, second parameter the same
\end{enumerate}

\textbf{Case both parameters are positive:} \\
If both parameters are positive, the modulo value of both values by 2 is 1.
Therefore the very last line of the function is evaluated. For decoded
values the result is $a+b$, which equals to $\pi(a) + \pi(b)$ for encoded
values as far as the sum of two even numbers is even (which represents
a positive number).

\end{document}
