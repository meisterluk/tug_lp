\documentclass[a4paper,twocolumn]{article}
\usepackage[utf8]{inputenc}

\author{Lukas Prokop}
\title{Wahrscheinlichkeitstheorie für Informatikstudien}

\begin{document}
\maketitle

\begin{description}
  \item[$\Omega$] Grundmenge (mögliche Ausgänge)
  \item[$\mathcal{A}$] Ereignisraum
  \item[$(\Omega, \mathcal{A})$] Stichprobenraum
  \item[$\{w_i\}$] Elementarereignisse
    (Ausgänge eines Zufallsexperiments)
  \item[$A (\in \mathcal{A})$] Ereignis
  \item[Statistische Regularität] Gesetz der großen Zahlen
  \item[Population, Grundgesamtheit]
    mögliche Ereignisse und ihre Wahrscheinlichkeit
  \item[Stichproben] Teilmengen der Population
  \item[Wahrscheinlichkeitsraum] \hfill{} \\
    $\{\Omega, A, P\}, A \leftarrow P(A),
        A (\in \mathcal{A}) \subseteq \Omega$
  \item[$\sigma$-Algebra] \hfill{} \\
    $\mathcal{A} \subseteq P(\Omega)$ \\
    $\Omega \in \mathcal{A}$, $A \in \mathcal{A} \Rightarrow \bar{A} \in
    \mathcal{A}$ \\
    $A_n \in \mathcal{A} \Rightarrow \bigcup_{n=1}^\infty A_n \in
    \mathcal{A}$ \\
    \dots ist $\sigma$-Algebra bei $\Omega \neq \emptyset$
  \item[C] Combination
  \item[V] Variation
\end{description}

\[
    \frac{\mbox{günstig}}{\mbox{möglich}} \Rightarrow
        \frac{\mbox{Maß(A)}}{\mbox{Maß}(\Omega)}
\] \[
    P(A\cup B) = P(A) + P(B)
\]

In LAPLACE Wahrscheinlichkeitsräumen reduziert sich die Berechnung der
günstigen Fälle mit ihrer Wahrscheinlichkeit auf kombinatorische
Zählprobleme.

\[
    P(E) = \frac{| E |}{| \Omega |}
\] \[
    {n \choose k} = \frac{n!}{(n-k)!\cdot k!}
        \hspace{15pt} 0 \leq k \leq n
\]

\begin{description}
  \item[r-Variation mit WH]
    Aus n Elementen wird r-mal mit Zurücklegen gezogen \\
    \emph{python:} itertools.product(n, repeat=r)
  \item[Permutation]
    r-Variation ohne WH mit $r=n$. \\
    \emph{python:} itertools.permutations(n, r)
  \item[r-Kombination mit WH] Anordnungsproblem:
    r nicht unterscheidbare Bälle werden in n numerierte Zellen gelegt
  \item[r-Kombination ohne WH] \hfill{} \\
    \emph{python:} itertools.combinations(n, r) \\
  \item[Wiederholung (WH)]
    $w_i = w_{i+n}$ mit $n>0$
  \item[Geordnet] $(w_1, w_2, w_3) \neq (w_1, w_3, w_2)$
\end{description}

\subsection{Beispiel für $n=\{1,2,3\}, r=2$}

\begin{tabular}{c|c|c}
                &       mit WH        &      ohne WH      \\
    \hline
    V           & $\begin{array}{ccc}
                  (1,1) & (1,2) & (1,3) \\
                  (2,1) & (2,2) & (2,3) \\
                  (3,1) & (3,2) & (3,3) \\
                  \multicolumn{3}{c}{V_w(n,r) = n^r} \\
                  \end{array}$
                                      & $\begin{array}{ccc}
                                        (1,2) & (1,3) & (2,1) \\
                                        (2,3) & (3,1) & (3,2) \\
                                        \multicolumn{3}{c}{
                                            V(n,r) = \frac{n!}{(n-r)!}} \\
                                        \end{array}$ \\
    \hline
    C           & $\begin{array}{ccc}
                  (1,1) & (1,2) & (1,3) \\
                  (2,2) & (2,3) & \\
                  (3,3) &       & \\
                  \multicolumn{3}{c}{C_w(n,r) = {n+r-1 \choose r}} \\
                  \end{array}$
                                      & $\begin{array}{ccc}
                                        (1,2) & (1,3) & (2,3) \\
                                        \multicolumn{3}{c}{C(n,r)
                                            = {n \choose r}}
                                        \end{array}$ \\
\end{tabular}

\end{document}
